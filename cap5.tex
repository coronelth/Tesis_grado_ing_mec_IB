\chapter{Conclusiones y próximos pasos}\label{cap5}
\chapterquote{Amigo Edison. Solo conozco a dos personas que merezcan llamarse “genio”. Una es usted; la otra es el joven que porta esta carta}{Carta de recomendación de Nikola Tesla.}

En el presente trabajo se estudiaron algunos aspectos fundamentales de la Teoría de Campos en el límite no relativista, y hacia el final, introduciendo una densidad finita de materia.

En el segundo capítulo se obtuvo una acción efectiva a bajas energías partiendo del campo relativista de Klein-Gordon. Con ella se determinaron las soluciones y la teoría se cuantizó siguiendo el esquema canónico. Se estudiaron estados de partículas de dos tipos: por un lado localizadas en el espacio de momentos  y por el otro, localizadas en el espacio real. Éste último tipo de partículas son inexistentes en la teoría relativista original. También se calculó el propagador a partir del formalismo operatorial, y se discutió la pérdida de la noción de microcausalidad debido a que la máxima velocidad de propagación no se encuentra acotada.

En el tercer capítulo, se estudió el Teorema de Reeh-Schlieder. Se analizó su demostración en el caso relativista, y en particular, el corolario que implica la separabilidad del vacío. Por otro lado, se extendió la demostración del teorema al límite no relativista, aunque en este caso la separabilidad del vacío no ocurría debido a la falta de conmutatividad de los campos a tiempos distintos. Además, se introdujo la noción de entropía de entrelazamiento y se concluyó que la misma es cero para cualquier región en el límite no relativista. La validez del teorema de Reeh-Schlider y la nulidad de la entropía de entrelazamiento a priori hubiesen parecido incompatibles, sin embargo la existencia de estados de partículas localizadas en posición y la existencia de traslaciones espacio-temporales en el límite no relativista, permiten que ambas proposiciones puedan coexistir.



Finalmente, en el cuarto capítulo se consideró un campo de Dirac libre discretizado en 1+1 dimensiones espacio-temporales con una densidad no trivial de materia. Se realizaron cálculos para determinar la función de correlación de dos puntos, que restringida a una región permite obtener la entropía de entrelazamiento. En particular se observó que para $|\mu|\leq m$ la entropía continua siendo la misma que para $\mu=0$. Por otro lado, se realizó un programa que por el momento calcula los valores de la entropía de entrelazamiento de un intervalo en la teoría discretizada. Para $m=0.1$ y $\mu=0.11$ se ajustó una dependencia logarítmica y se obtuvo una constante multiplicativa de $0,343$ que coincide aproximadamente con el término universal de $1/3$ que aparece en sistemas en 1+1 dimensiones. 

Los próximos pasos consistirán en obtener las curvas de entropía de entrelazamiento que sean representativas de la teoría en el continuo. En dicha tarea deben modificarse dos páramentros adimensionales, por tal motivo se requiere optimizar al máximo el código numérico  antes de emplearlo. Una vez obtenidas las curvas, se esperarían apreciar distintos regímenes, entre ellos uno en el cual la entropía sea constante como en el límite no relativista. También, se buscarán utilizar otros indicadores como la entropía relativa para cuantificar como el vacío va “volviendose separable” al prender un potencial químico. En un período posterior, el objetivo del trabajo consistirá en estudiar el mismo sistema pero en 2+1 dimensiones espacio-temporales. En tal caso, se espera la presencia de efectos más dramáticos debido a la existencia de la superficie de Fermi. Esto se debe a que en $d=3$ para una CFT, la entropía de una bola escalea con la ley de áreas, es decir $S(R)\sim \frac{R}{\epsilon}$ (contribución UV divergente), mientras que debido a la presencia del potencial químico, la entropía debería escalear como $S(R)\sim (k_F R) \log(k_F R)$ (contribución finita). Por lo tanto, esto conduciría a una violación logarítmica de la ley de áreas, efecto no apreciable en $d=2$.

%%% Local Variables:
%%% mode: latex
%%% TeX-master: "template"
%%% End:
