\chapter{Introducci\'on y motivaci\'on}\label{cap1}
\chapterquote{No deseo estatuas, placas, premios, calles o institutos cuando muera. Mis esperanzas son otras. Deseo que mi país contribuya al adelanto científico y cultural del mundo científico actual. Que tenga artistas, pensadores y científicos que enriquezcan nuestra cultura y cuya obra sea beneficiosa para nuestro país, nuestros compatriotas y toda la especie humana.}{Bernardo Houssay. Premio Nobel en Fisiología (1947)}

En el estudio de la Mecánica de los Fluidos, es de importancia los problemas de transferencia de calor de flujos multi-fásicos con cambios de fase. Para dichos problemas cuyas aplicaciones industriales entre otras son la transferencia de calor  que se produce en las barras de elementos combustibles del núcleo de un reactor nuclear. Se realizan mediciones de las variables físicas involucradas en el estudio de dicho problema, en base a las mediciones se realizan modelos para explicar la fenomenología. Lo que no se posee en la actualidad es un modelo que se pueda simular numéricamente y que a su vez posea un bajo costo computacional para realizarlo. El estudio de las variables físicas  

La mecánica de los fluidos se puede describir en tres niveles: macroscópica, mesoscópica y microscópica. En el nivel macroscópico, predominan las leyes físicas de conservación de la masa, momento y energía aplicada a un volumen de control establecidas por un conjunto de ecuaciones diferenciales (ecuaciones de masa, momento y energía) que gobiernan el comportamiento del fluido. La Mecánica de Fluidos Computacional (\textit{Computational Fluid Dynamics} o CFD) es utilizada para resolver esas ecuaciones que gobiernan el comportamiento físico utilizando distintos métodos numéricos. En contraste el método de lattice Boltzmann(LBM) es una aproximación del nivel mesoscópica. LBM estudia la micro-dinámica de partículas ficticias utilizando modelos cinéticos simplificados. La cual provee un camino alternativo de simular la mecánica de fluidos. La naturaleza de la cinética brinda distintas características de LBM tales que es claro el panorama de los procesos de advección y colisión de partículas de fluidos simuladas; la estructura simple del algoritmo, la fácil implementación de condiciones de contorno y el natural paralelismo. Todos éstos interesantes atributos hacen que LBM sea una potente herramienta numérica para la simulación de sistemas de fluidos envueltos en problemas físicos complejos.

Los fluidos como el aire y el agua son frecuentemente conocidos en nuestra vida diaria. Físicamente todos los fluidos son compuestos de un gran conjunto de átomos o moléculas que chocan unas con otras moviéndose  aleatoriamente. Interacciones de moléculas en un fluido son usualmente más débiles que las mismas en un sólido y un fluido puede ser deformado continuamente bajo una pequeña aplicación de esfuerzo. Usualmente la dinámica microscópica de las moléculas del fluido son muy complicadas y demuestran una fuerte inhomogeneidad y fluctuaciones. Por el otro lado la dinámica macroscópica del fluido el cual es el resultado medio del movimiento de las moléculas en un medio homogéneo y continuo.  También puede ser explicado mediante modelos matemáticos de la dinámica de los fluidos una fuerte dependencia del largo y el tamaño de las escalas y cuál es el fluido observado. Generalmente el movimiento de un fluido puede ser descripto por tres tipos de modelos matemáticos acuerdo a lo que se observan en las distintas escalas, por ejemplo microscópico en modelos de escala molecular, teorías cinéticas en la escala mesoscópica y modelos continuos para escalas macroscópicas.

Los modelos matemáticos de los flujos de fluidos, como también las ecuaciones de Newton para un basto número de moléculas, o las ecuaciones de Boltzman para la función de e distribución simple o las ecuaciones de Navier-Stokes para las variaciones de flujos macroscópicos, son extremadamente difíciles de resolver analíticamente de no ser imposibles. La precisión de los modelos numéricos sin embargo han provisto de manera satisfactoria soluciones aproximadas de dichas ecuaciones. Particularmente con la rapidez del desarrollo del software y hardware computacional y la tecnología, las simulaciones numéricas han comenzado a ser una importante metodología para la dinámica de fluidos.

El más exitoso y popular método de simulación de fluidos es la tećnica CFD , el cual su principal diseño está basado en  para resolver ecuaciones hidrodinámicas basadas en supociones de continuidad. En CFD el dominio del flujo está compuesto en en conjunto de de sub-dominios con una malla computacional, , las ecuaciones matemáticas son discretizadas usando algunos esquemas de discretización numérica como elementos finitos, volúmenes finitos o diferencia finitas; los cuales resultan en un sistema algebraico de sistemas de ecuaciones para las variables discretas del fluido asociadas a la malla computacional. Computacionalmente son llevados a cabo para encontrar una solución aproximada resolviendo el problema algebraico de sistemas de ecuaciones usando un algoritmo secuencial o paralelo.

La Simulación de Dinámica Molecular (\textit{Molecular Dynamic Simulation} o MDS) es una técnica en la cual el movimiento individual de átomos o moléculas del fluido son registrados para resolver las ecuaciones de Newton mediante una computadora. La principal ventaja de MDS es el aspecto macroscópico del fluido puede ser directamente conectado con el comportamiento molecular, en donde la estructura molecular y las interacciones microscópicas pueden ser descriptas de una manera directa.




La Teoría Cuántica de Campos (\textit{Computational Fluid Dynamics} o CFD) describe sistemas cuánticos con infinitos grados de libertad, y juega un rol central en modelos de altas energías y sistemas fuertemente correlacionados en materia condensada. El denominado grupo de renormalización (RG) propone reorganizar la Teoría Cuántica de Campos en términos de acciones efectivas que cambian con la escala de energía, y busca encontrar los grados de libertad que dominan los procesos cuánticos a una dada escala\cite{Wilson:1973jj}. El flujo del grupo de renormalización entre puntos fijos, es decir entre las teorías de campos conformes (CFT) en el UV (ultravioleta) y en el IR (infrarojo), está fuertemente ligado con la denominada entropía de entrelazamiento de una región. La misma es una medida de las correlaciones del vacío de la teoría entre una región del espacio $V$ y su complemento $\bar{V}$\cite{Casini:2009sr}. Utilizando la subaditividad fuerte y la invarianza de Lorentz se ha demostrado que el término universal constante del desarrollo en potencias o logarítmico (para dimensiones pares) de la entropía de entrelazamiento, tomando como región una una ($d\:$-1)-esfera, para $d=2,3\text{ y }4$ decrece debido al flujo del grupo de renormalización\cite{Zamolodchikov:1986gt}\cite{Casini:2012ei}\cite{Casini:2017vbe}. Estos teoremas se denominan de irreversibilidad y representan una generalización cuántica de la irreversibilidad de la entropía termodinámica. Aparentemente, existen modelos en donde la invarianza de Lorentz se encuentra ausente y parecerían violarse dichas desigualdades \citep{Swingle:2014}.

En el marco de la Teoría Axiomática de Campos, un resultado conocido como Teorema de Reeh-Schlieder es sorprendente en términos de causalidad\cite{Wightman:2016}. En particular, uno de sus corolarios predice la existencia de estados con energía negativa y además, la existencia de correlaciones del vacío para regiones que poseen una separación de tipo espacio, es decir, predice el entrelazamiento del vacío para la teoría relativista\cite{Witten:2018lha}. Naturalmente resulta de interés si éstos resultados siguen siendo válidos aun en la teoría no relativista en donde los estados poseen energía no negativa y además se ha demostrado que el vacío de la teoría no se encuentra entrelazado\cite{Hason:2017flq}.\\

En cuanto a la estructura de la tesis, la misma constará de tres capítulos de contenidos además de una sección con conclusiones y miras hacia la continuación del trabajo: 
\begin{itemize}
\item  En el capítulo 2, se estudiará como obtener una acción efectiva no relativista a partir de la acción de Klein-Gordon. Se cuantizará la nueva teoría y se demostrará la existencia de estados de partículas localizadas en el espacio. Este último resultado será clave, como se verá en el Capítulo 3, para entender la nulidad de la entropía de entrelazamiento en la teoría no relativista. Además, se calculará el propagador y se discutirán nociones de causalidad.
\item En el capítulo 3, se estudiará el Teorema de Reeh-Schlieder y se revisará su demostración para el caso relativista. Se hará una extensión de la misma al caso no relativista y se discutirá la falla de sus corolarios en dicho régimen. También se introducirá formalmente la noción de entropía de entrelazamiento y se dará una prueba de su nulidad en la teoría no relativista. Ese resultado podrá entenderse debido a la existencia de estados de partículas localizadas en el espacio.
\item Por último, en el capítulo 4 se estudiará la entropía de entrelazamiento del campo de Dirac libre en 1+1 dimensiones espacio-temporales en presencia de una densidad no trivial de materia, es decir, con un potencial químico $\mu \neq 0$. Esto último con el objetivo de comprobar uno de los resultados que parecerían violar los teoremas de irreversibilidad en \citep{Swingle:2014}. Siguiendo los pasos de \cite{Casini:2009sr} y \cite{peschel2002} se calculará la función de correlación de dos puntos, que restringida a una región $V$, permite calcular la entropía de entrelazamiento dado que la teoría es libre. Para ello, se realizó un programa que computa la entropía de entrelazamiento en el discreto de la teoría. Queda aun el trabajo de obtener los valores de la entropía en el continuo de la teoría.
\end{itemize}
  


%%% Local Variables:
%%% mode: latex
%%% TeX-master: "template"
%%% End:
