
%%%%%%%%%%%%%%%%%%%%%%%%%%%%%%%%%%%%%%%%%%%%%%%%%%%%%%%%%%%%%%%%%%%%%%%%%%%%%%%%
%\documentclass[12pt,papel,twoside]{ibtesis}
\documentclass[12pt,screen,oneside,pagebackref]{ibtesis}
%\documentclass[12pt,screen,twoside,pagebackref]{ibtesis}
%\documentclass[12pt,papel,singlespace,oneside]{ibtesis}
%\documentclass[12pt,papel,preprint,singlespace,oneside]{ibtesis}

%%%%%%%%%%%%%%%%%%%%% Paquetes extra %%%%%%%%%%%%%%%%%%%%%%%%%%%%%%%%%%%%%%%%%%%
% Por conveniencia: aqu\'{\i} puede cargar todos los paquetes y definir los comandos
% que necesite
%%%%%%%%%%%%%%%%%%%%% Paquetes extra %%%%%%%%%%%%%%%%%%%%%%%%%%%%%%%%%%%%%%%%%%%
% Por conveniencia: aqu\'{\i} puede cargar todos los paquetes y definir los comandos
% que necesite
\usepackage[utf8]{inputenc}
%\usepackage{ibextra}
\usepackage{amsmath}
\usepackage{amssymb}
\usepackage[spanish]{babel}
\usepackage{braket}
\usepackage{listings}
\usepackage{mathrsfs}
%\usepackage{graphicx}
\usepackage{subcaption}
%\usepackage{float}
\usepackage{wrapfig}
\usepackage{bbm}

%\newtheorem{theorem}{Teorema}[section]
%\newtheorem{corollary}{Corollary}[theorem]
%\newtheorem{lemma}[theorem]{Lema}

\DeclareMathOperator{\Tr}{Tr}
\newcommand{\Lagr}{\mathcal{L}}
\newcommand\p[2]{\frac{\partial #1}{\partial #2}}
\renewcommand{\Im}{\mathop{\rm Im}\nolimits}
\def\e{{\rm e}}
\hyphenation{con-si-de-ra-mos}

%%%%%%%%%%%%%%%%%%%%%%%%%%%%%%%%%%%%%%%%%%%%%%%%%%%%%%%%%%%%%%%%%%%%%%%%%%%%%%%%
%%%%%%%%%%%%%%%%%%%%% Informacion sobre la tesis %%%%%%%%%%%%%%%%%%%%%%%%%%%%%%%
\title{Implementaci\'on de modelos lattice-Boltzmann para flujo multif\'asico con transferencia de calor en unidades de procesamiento gr\'afico}
\author{Thom\'as Coronel}
\director{Mgter. Ezequiel Fogliatto}
\codirector{Ing. Pablo Argañaraz}
\carrera{Tesis Carrera de Ingenier\'ia Mec\'anica}
\grado{}
\laboratorio{Mecánica Computacional - Centro Atómico Bariloche}

\jurado{ Dr. René Cejas Bolecek (Instituto Balseiro - Centro Atómico Bariloche)}
%\jurado{ Dr. Flavio Colavecchia (Instituto Balseiro - Centro Atómico Bariloche)}


%\palabrasclave{Espectroscop\'ia de fotoemisi\'on resuelta en tiempo, respuesta electr\'onica ultrarr\'apida, ``pump and probe'', ARPES, teor\'ia de Floquet, grafeno, aislantes topol\'ogicos.}
%\keywords{Time resolved photoemission spectroscopy, electronic ultrafast response, pump and probe, Floquet theory, graphene, topological insulators.}
% Si queremos poner la fecha manualmente:
\date{Junio de 2020}

%%%%%%%%%%%%%%%%%%%%%%%%%%%%%%%%%%%%%%%%%%%%%%%%%%%%%%%%%%%%%%%%%%%%%%%%%%%%%%%%
%\titlepagefalse % Si no quiere compilar la portada descomente esta linea
%\includeonly{apendices} % Compilar s\'{o}lo estos archivos
\graphicspath{{figs/}} % Lugar donde encontrar las figuras generales (se puede poner uno en cada cap{\'{\i}}tulo)
%%%%%%%%%%%%%%%%%%%%%%%%%%%%%%%%%%%%%%%%%%%%%%%%%%%%%%%%%%%%%%%%%%%%%%%%%%%%%%%%


\begin{document}

% Dentro del environment 'preliminary' va:
% la dedicatoria, resumen, abstract, indices


\begin{preliminary}
% Escriba su dedicatoria

\dedicatoria{\textit{Inserte aquí su dedicatoria.}}


%%% \'{I}ndices %%%%

%\begin{abreviaturas}
%Abreviaturas
%\end{abreviaturas}

\tableofcontents                %\'{I}ndice

%\listoffigures                  %Figuras

%\listoftables                   %Tablas

%\chapter{Resumen}\label{resumen}

Te lo resumo así nomás
%\chapter{Abstract}\label{abstract}

The abstract boy

\end{preliminary}

% Podemos usar cualquiera de los dos comandos: \input o \include para incluir el texto
%\include{cap0}
\chapter{Introducci\'on y motivaci\'on}\label{cap1}
\chapterquote{No deseo estatuas, placas, premios, calles o institutos cuando muera. Mis esperanzas son otras. Deseo que mi país contribuya al adelanto científico y cultural del mundo científico actual. Que tenga artistas, pensadores y científicos que enriquezcan nuestra cultura y cuya obra sea beneficiosa para nuestro país, nuestros compatriotas y toda la especie humana.}{Bernardo Houssay. Premio Nobel en Fisiología (1947)}

En el estudio de la Mecánica de los Fluidos, es de importancia los problemas de transferencia de calor de flujos multi-fásicos con cambios de fase. Para dichos problemas cuyas aplicaciones industriales entre otras son la transferencia de calor  que se produce en las barras de elementos combustibles del núcleo de un reactor nuclear. Se realizan mediciones de las variables físicas involucradas en el estudio de dicho problema, en base a las mediciones se realizan modelos para explicar la fenomenología. Lo que no se posee en la actualidad es un modelo que se pueda simular numéricamente y que a su vez posea un bajo costo computacional para realizarlo. El estudio de las variables físicas  

La mecánica de los fluidos se puede describir en tres niveles: macroscópica, mesoscópica y microscópica. En el nivel macroscópico, predominan las leyes físicas de conservación de la masa, momento y energía aplicada a un volumen de control establecidas por un conjunto de ecuaciones diferenciales (ecuaciones de masa, momento y energía) que gobiernan el comportamiento del fluido. La Mecánica de Fluidos Computacional (\textit{Computational Fluid Dynamics} o CFD) es utilizada para resolver esas ecuaciones que gobiernan el comportamiento físico utilizando distintos métodos numéricos. En contraste el método de lattice Boltzmann(LBM) es una aproximación del nivel mesoscópica. LBM estudia la micro-dinámica de partículas ficticias utilizando modelos cinéticos simplificados. La cual provee un camino alternativo de simular la mecánica de fluidos. La naturaleza de la cinética brinda distintas características de LBM tales que es claro el panorama de los procesos de advección y colisión de partículas de fluidos simuladas; la estructura simple del algoritmo, la fácil implementación de condiciones de contorno y el natural paralelismo. Todos éstos interesantes atributos hacen que LBM sea una potente herramienta numérica para la simulación de sistemas de fluidos envueltos en problemas físicos complejos.

Los fluidos como el aire y el agua son frecuentemente conocidos en nuestra vida diaria. Físicamente todos los fluidos son compuestos de un gran conjunto de átomos o moléculas que chocan unas con otras moviéndose  aleatoriamente. Interacciones de moléculas en un fluido son usualmente más débiles que las mismas en un sólido y un fluido puede ser deformado continuamente bajo una pequeña aplicación de esfuerzo. Usualmente la dinámica microscópica de las moléculas del fluido son muy complicadas y demuestran una fuerte inhomogeneidad y fluctuaciones. Por el otro lado la dinámica macroscópica del fluido el cual es el resultado medio del movimiento de las moléculas en un medio homogéneo y continuo.  También puede ser explicado mediante modelos matemáticos de la dinámica de los fluidos una fuerte dependencia del largo y el tamaño de las escalas y cuál es el fluido observado. Generalmente el movimiento de un fluido puede ser descripto por tres tipos de modelos matemáticos acuerdo a lo que se observan en las distintas escalas, por ejemplo microscópico en modelos de escala molecular, teorías cinéticas en la escala mesoscópica y modelos continuos para escalas macroscópicas.

Los modelos matemáticos de los flujos de fluidos, como también las ecuaciones de Newton para un basto número de moléculas, o las ecuaciones de Boltzman para la función de e distribución simple o las ecuaciones de Navier-Stokes para las variaciones de flujos macroscópicos, son extremadamente difíciles de resolver analíticamente de no ser imposibles. La precisión de los modelos numéricos sin embargo han provisto de manera satisfactoria soluciones aproximadas de dichas ecuaciones. Particularmente con la rapidez del desarrollo del software y hardware computacional y la tecnología, las simulaciones numéricas han comenzado a ser una importante metodología para la dinámica de fluidos.

El más exitoso y popular método de simulación de fluidos es la tećnica CFD , el cual su principal diseño está basado en  para resolver ecuaciones hidrodinámicas basadas en supociones de continuidad. En CFD el dominio del flujo está compuesto en en conjunto de de sub-dominios con una malla computacional, , las ecuaciones matemáticas son discretizadas usando algunos esquemas de discretización numérica como elementos finitos, volúmenes finitos o diferencia finitas; los cuales resultan en un sistema algebraico de sistemas de ecuaciones para las variables discretas del fluido asociadas a la malla computacional. Computacionalmente son llevados a cabo para encontrar una solución aproximada resolviendo el problema algebraico de sistemas de ecuaciones usando un algoritmo secuencial o paralelo.

La Simulación de Dinámica Molecular (\textit{Molecular Dynamic Simulation} o MDS) es una técnica en la cual el movimiento individual de átomos o moléculas del fluido son registrados para resolver las ecuaciones de Newton mediante una computadora. La principal ventaja de MDS es el aspecto macroscópico del fluido puede ser directamente conectado con el comportamiento molecular, en donde la estructura molecular y las interacciones microscópicas pueden ser descriptas de una manera directa.




La Teoría Cuántica de Campos (\textit{Computational Fluid Dynamics} o CFD) describe sistemas cuánticos con infinitos grados de libertad, y juega un rol central en modelos de altas energías y sistemas fuertemente correlacionados en materia condensada. El denominado grupo de renormalización (RG) propone reorganizar la Teoría Cuántica de Campos en términos de acciones efectivas que cambian con la escala de energía, y busca encontrar los grados de libertad que dominan los procesos cuánticos a una dada escala\cite{Wilson:1973jj}. El flujo del grupo de renormalización entre puntos fijos, es decir entre las teorías de campos conformes (CFT) en el UV (ultravioleta) y en el IR (infrarojo), está fuertemente ligado con la denominada entropía de entrelazamiento de una región. La misma es una medida de las correlaciones del vacío de la teoría entre una región del espacio $V$ y su complemento $\bar{V}$\cite{Casini:2009sr}. Utilizando la subaditividad fuerte y la invarianza de Lorentz se ha demostrado que el término universal constante del desarrollo en potencias o logarítmico (para dimensiones pares) de la entropía de entrelazamiento, tomando como región una una ($d\:$-1)-esfera, para $d=2,3\text{ y }4$ decrece debido al flujo del grupo de renormalización\cite{Zamolodchikov:1986gt}\cite{Casini:2012ei}\cite{Casini:2017vbe}. Estos teoremas se denominan de irreversibilidad y representan una generalización cuántica de la irreversibilidad de la entropía termodinámica. Aparentemente, existen modelos en donde la invarianza de Lorentz se encuentra ausente y parecerían violarse dichas desigualdades \citep{Swingle:2014}.

En el marco de la Teoría Axiomática de Campos, un resultado conocido como Teorema de Reeh-Schlieder es sorprendente en términos de causalidad\cite{Wightman:2016}. En particular, uno de sus corolarios predice la existencia de estados con energía negativa y además, la existencia de correlaciones del vacío para regiones que poseen una separación de tipo espacio, es decir, predice el entrelazamiento del vacío para la teoría relativista\cite{Witten:2018lha}. Naturalmente resulta de interés si éstos resultados siguen siendo válidos aun en la teoría no relativista en donde los estados poseen energía no negativa y además se ha demostrado que el vacío de la teoría no se encuentra entrelazado\cite{Hason:2017flq}.\\

En cuanto a la estructura de la tesis, la misma constará de tres capítulos de contenidos además de una sección con conclusiones y miras hacia la continuación del trabajo: 
\begin{itemize}
\item  En el capítulo 2, se estudiará como obtener una acción efectiva no relativista a partir de la acción de Klein-Gordon. Se cuantizará la nueva teoría y se demostrará la existencia de estados de partículas localizadas en el espacio. Este último resultado será clave, como se verá en el Capítulo 3, para entender la nulidad de la entropía de entrelazamiento en la teoría no relativista. Además, se calculará el propagador y se discutirán nociones de causalidad.
\item En el capítulo 3, se estudiará el Teorema de Reeh-Schlieder y se revisará su demostración para el caso relativista. Se hará una extensión de la misma al caso no relativista y se discutirá la falla de sus corolarios en dicho régimen. También se introducirá formalmente la noción de entropía de entrelazamiento y se dará una prueba de su nulidad en la teoría no relativista. Ese resultado podrá entenderse debido a la existencia de estados de partículas localizadas en el espacio.
\item Por último, en el capítulo 4 se estudiará la entropía de entrelazamiento del campo de Dirac libre en 1+1 dimensiones espacio-temporales en presencia de una densidad no trivial de materia, es decir, con un potencial químico $\mu \neq 0$. Esto último con el objetivo de comprobar uno de los resultados que parecerían violar los teoremas de irreversibilidad en \citep{Swingle:2014}. Siguiendo los pasos de \cite{Casini:2009sr} y \cite{peschel2002} se calculará la función de correlación de dos puntos, que restringida a una región $V$, permite calcular la entropía de entrelazamiento dado que la teoría es libre. Para ello, se realizó un programa que computa la entropía de entrelazamiento en el discreto de la teoría. Queda aun el trabajo de obtener los valores de la entropía en el continuo de la teoría.
\end{itemize}
  


%%% Local Variables:
%%% mode: latex
%%% TeX-master: "template"
%%% End:

\include{ecu}
%\chapter{L\'imite no relativista del campo escalar real}\label{cap2}
\chapterquote{Hay hombres que de su cencia\\
tienen la cabeza llena;\\
hay sabios de todas menas,\\
mas digo, sin ser muy ducho,\\  
es mejor que aprender mucho \\
el aprender cosas buenas.}{José Hernández. La vuelta del Martín Fierro.}

En el marco de la Teoría Cuántica de Campos (QFT) un objeto de relevante interés es el campo de Klein-Gordon $\hat{\phi}(x^{\mu})$ que describe el comportamiento de partículas relativistas de \textit{spin} 0. El objetivo de este capítulo radica en entender como partiendo del formalismo relativista de Klein-Gordon se puede obtener una teoría cuántica no relativista, en donde la ecuación de Schr\"{o}dinger domine la evolución temporal. Particularmente se estudiará la noción de estados localizados en momento y en posición, y se concluirá que estos últimos sólo están bien definidos en la teoría no relativista. También se calculará el propagador de la teoría no relativista a partir de las soluciones del campo, y se discutirá la noción de microcausalidad.

\section{Campo escalar real clásico}
Consideremos un campo “clásico”\footnote{Se lo denomina “clásico” ya que todavía no se han introducido ingredientes inherentes al mundo cuántico.} $\phi(x^{\mu})$, tal que sea una función escalar y real, es decir, $\phi: \mathbb{R}^{d+1}\mapsto \mathbb{R}$ y que satisfaga la ecuación de Klein-Gordon\footnote{Se utilizará la métrica Lorentziana $g_{\mu \nu}=(+-\ldots -)$ con $d+1$ coordenadas espacio-temporales.}
\begin{equation}
(\square+m^2)\phi(x^{\mu})=(\partial^{\nu}\partial_{\nu}+m^2)\phi(x^{\mu})=0.
\label{eq:KG}
\end{equation}
Una posible acción $S$ tal que al minimizarla $\delta S=0$ se obtenga (\ref{eq:KG}) resulta
\begin{equation}
S = \int dt d^d x\, \frac{1}{2} \left[ (\partial_\mu \phi)^2 - m^2 \phi^2\right]\,.
\label{eq:action_KG}
\end{equation}
La condición en capa de masa establece que la energía $E$ en términos del momento $\vec{p}$ de cada modo de Fourier del campo viene dada por
\begin{equation}
\omega_{\vec{p}}\equiv E=\pm \sqrt{p^2+m^2}.
\end{equation}
En el límite en el que $p \ll m$,
\begin{equation}
\omega _{\vec{p}}= \pm \left(m + \frac{p^2}{2m}+ \mathcal{O}\left(\frac{p^4}{m^3}\right)\right)\,.
\end{equation}
La energía posee un orden dominante $\omega_{\vec{p}}\sim \pm m$ más pequeñas correciones entre las cuales aparece la relación usual no relativista $\frac{p^2}{2m}$. Para desacoplar los modos menos energéticos, definimos un campo escalar complejo $\psi:\mathbb{C}^{d+1}\mapsto \mathbb{C}$ tal que\footnote{En realidad, la expresión (\ref{eq:cambio_psi}) tendría que escribirse en términos de $\psi$ y $\psi^{*}$. Sin embargo, al cuantizar el campo $\psi$ y $\psi^{*}$ $\mapsto$ $\hat{\psi}$ y $\hat{\psi^{\dag}}$, con lo cual se optará por usar sistemáticamente $\dag$ en vez de $*$.} 
\begin{equation}
\phi \equiv \frac{1}{\sqrt{2m}} \left(e^{-imt} \psi + e^{i m t} \psi^\dag \right).
\label{eq:cambio_psi}
\end{equation}
Al reemplazar en la acción (\ref{eq:action_KG}) y despreciando los términos que oscilan rápidamente debido a los factores $e^{\pm i2mt}$, se obtiene que
\begin{equation}
S\approx \int dt d^dx\,\left (\frac{i}{2} (\psi^\dag \partial_t \psi - \partial_t \psi^\dag\,\psi)- \frac{1}{2m}|\partial_i \psi|^2+\frac{1}{2m} |\partial_t \psi|^2 \right)\,.
\end{equation}
Como el último término $\frac{1}{2m} |\partial_t \psi|^2$ está suprimido por la masa en comparación con los primeros dos $\frac{i}{2} (\psi^\dag \partial_t \psi - \partial_t \psi^\dag\,\psi)$, también podemos despreciarlo y arribar a una nueva acción aproximada
\begin{equation}
S_{\text{NR}} = \int dt d^dx\,\left (\frac{i}{2} (\psi^\dag \partial_t \psi - \partial_t \psi^\dag\,\psi)- \frac{1}{2m}|\partial_i \psi|^2 \right)\,.
\label{eq:action_sch1}
\end{equation}
La acción (\ref{eq:action_sch1}) usualmente se la denomina la acción de Schr\"{o}dinger pues al minimizar acción $\psi$ y $\psi^{\dag}
$ satisfacen la ecuación homónima y su conjugada en ausencia de potencial
\begin{equation}
i\partial_t \psi=-\frac{\nabla^2}{2m}\psi \:\:\:\:\:\: \text{y} \:\:\:\:\: i\partial_t \psi^{\dag}=\frac{\nabla^2}{2m}\psi^{\dag}. 
\end{equation}
Integrando por partes y despreciando los términos de borde la acción (\ref{eq:action_sch1}) puede reescribirse como usualmente aparece en la literatura
\begin{equation}
S_{\text{NR}}=\int dt d^dx\,\underbrace{\psi^\dag \left(i\partial_t +\frac{\partial_i^2}{2m}   \right)\psi}_{\mathcal{L}_{\text{NR}}} \,  .
\end{equation}
La acción no relativista posee una nueva simetría global U(1) dada por $\psi \mapsto e^{-i\alpha}\psi$, con $\alpha$ constante. Esta nueva simetría no existe en la teoría en el UV (ultravioleta), más bien, es una ley emergente de bajas energías dado que todos los términos que la rompen se encuentran suprimidos. Por el teorema de Noether, la corriente conservada $j^{\mu}$ asociada a esta simetría viene dada por
\begin{equation}
j^{0}=\frac{\partial \mathcal{L}_{\text{NR}}}{\partial(\partial_{0}\psi)}\frac{\delta \psi }{\delta \alpha}=\psi^{\dag}\psi \:\: \text{,}\:\: j^{i}=\frac{\partial \mathcal{L}_{\text{NR}}}{\partial(\partial_{i}\psi)}\frac{\delta \psi }{\delta \alpha}+\frac{\partial \mathcal{L}_{\text{NR}}}{\partial(\partial_{i}\psi^{\dag})}\frac{\delta \psi^{\dag} }{\delta \alpha}=\frac{i}{2m}(\psi \partial^i \psi^{\dag}- \psi^{\dag}\partial^i \psi).
\end{equation}
Mientras que la carga conservada $N$,
\begin{equation}
N=\int d^dx \:j^0=\int d^dx \:\psi^{\dag}\psi,
\end{equation}
resulta ser el número de partículas. En particular, esta magnitud es semidefinida positiva. Más adelante tendrá un rol fundamental a la hora de argumentar que la entropía de entrelazamiento en la teoría no relativista es cero.\\

Las soluciones de las ecuación de Schr\"{o}dinger en el espacio de momentos resultan ser ondas planas de momento $\vec{p}$ y energía $w_{\vec{p}}=\frac{p^2}{2m}$. Entonces, en el espacio real pueden escribirse como combinación lineal de ellas, pesadas por coeficientes complejos $a_{\vec{p}}$ y $a_{\vec{p}}^{\dag}$
\begin{equation}
\psi (\vec{x},t)=\int \frac{d^dp}{(2\pi)^d}\, a_{\vec{p}}\:e^{i(\vec{p}\cdot\vec{x}-w_{\vec{p}}\:t)},\,
\end{equation}
\begin{equation}
\psi ^{\dag}(\vec{x},t)=\int \frac{d^dp}{(2\pi)^d}\, a_{\vec{p}}^{\dag}\:e^{-i(\vec{p}\cdot\vec{x}-w_{\vec{p}}\:t)}.\,
\end{equation}
Por otro lado, el momento canónicamente conjugado respecto de $\psi$ es
\begin{equation}
\pi(\vec{x},t)=\frac{\partial \mathcal{L}_{\text{NR}}}{\partial(\partial_t \psi)}=i\psi^{\dag}(\vec{x},t).
\end{equation}

Con esto dicho, ya es posible continuar con miras hacia la cuantización del campo.
\section{Cuantización canónica}
Para conciliar los campos clásicos con la mecánica cuántica se puede seguir una prescripción denominada “cuantización canónica” en la cual los campos son elevados a operadores\footnote{Formalmente deberían indicarse tanto $\psi$, $a_{\vec{p}}$ y sus conjugados con un \^{} sobre ellos, aunque omitiremos su escritura de aquí en adelante.} actuando sobre un espacio de Fock y sobre los cuales se imponen reglas de conmutación denominadas canónicas:\footnote{En el caso no relativista también podrían imponerse relaciones de anticonmutación para describir partículas fermiónicas, sin embargo, si se tratasen de imponer en el campo de Klein-Gordon, se violarían principios básicos como por ejemplo microcausalidad.}
\begin{equation}
[\psi(\vec{x},t),\psi^{\dag}(\vec{x}\:',t)]=\delta^{(d)}(\vec{x}-\vec{x}\:')\:\:\:\:\:\: \text{y}\:\:\:\:\:\:[\psi(\vec{x},t),\psi(\vec{x}\:',t)]=[\psi^{\dag}(\vec{x},t),\psi^{\dag}(\vec{x}\:',t)]=0.
\label{eq:conmut_psi_psid}
\end{equation}
Puede demostrarse que las relaciones son equivalentes a
\begin{equation}
[a_{\vec{p}},a_{\vec{p}\:'}^{\dag}]=(2\pi)^d\delta^{(d)}(\vec{p}-\vec{p}\:')\:\:\:\:\:\:\text{y}\:\:\:\:\:\:[a_{\vec{p}},a_{\vec{p}\:'}]=[a_{\vec{p}}^{\dag},a_{\vec{p}\:'}^{\dag}]=0.
\end{equation}
En el formalismo del espacio de Fock se define el vacío $|0\rangle$ como el estado sin partículas tal que $\langle 0|0\rangle=1$ y $a_{\vec{p}}|0\rangle=0$ para todo $\vec{p}$. A continuación, veremos otros elementos que permitan interpretar a los estados del tipo $a_{\vec{p}}^{\dag}|0\rangle$.

El hamiltoniano $H$ puede escribirse en términos de la densidad hamiltoniana $\mathcal{H}=\pi\partial_t\psi-\mathcal{L}$ como
\begin{equation}
H=\int d^dx \, \mathcal{H} = \int \frac{d^dp}{(2\pi)^d}\, \omega_{\vec{p}}\:a_{\vec{p}}^{
\dag}\: a_{\vec{p}}.\,
\end{equation}
Como la densidad lagrangiana es invariante ante traslaciones espaciales, el momento $\vec{P}$ resulta ser una carga conservada (que en este formalismo también resulta ser un operador de campo),
\begin{equation}
(\vec{P})^i=\int d^dx \,T^{0i} \,=\int d^dx \,\pi\partial^i \psi \,=\int \frac{d^dp}{(2\pi)^d} \,(\vec{p})^i\:a_{p}^{\dag}a_{\vec{p}}\:, \,
\end{equation}
siendo $T^{\mu \nu}$ el tensor energía-impulso. Si definimos el vector $|\vec{p}\rangle$ como
\begin{equation}
|\vec{p}\rangle \equiv a_{\vec{p}}^{\dag}|0\rangle,
\end{equation}
puede demostrarse utilizando las expresiones de $H$ y $\vec{P}$ que: $H|\vec{p}\rangle = w_{\vec{p}} |\vec{p}\rangle $ y $\vec{P}|\vec{p}\rangle = \vec{p} |\vec{p}\rangle $. Esto último permite interpretar a los vectores $|\vec{p}\rangle $ como “partículas” de momento $\vec{p}$ y energía $w_{\vec{p}}$. Por este motivo se suele llamar a $a_{p}^{\dag}$ operador de creación y consecuentemente a $a_{\vec{p}}$ operador de destrucción. Sin embargo, estas “partículas” no se encuentran localizadas en el espacio. Por otro lado, sí existen vectores que se encuentran localizados en el espacio y que resultan ser combinación lineal de $|\vec{p}\rangle$:
\begin{equation}
|\vec{x}\rangle \equiv \psi^{\dag}(\vec{x},0)|0\rangle=\int \frac{d^dp}{(2\pi)^d} \,e^{-i\vec{p}\:\vec{x}}|\vec{p}\rangle \,.
\label{eq:local_nr}
\end{equation} 
Para demostrar esa última afirmación, es posible definir un operador posición $\vec{X}$ al igual que en mecánica cuántica no relativista
\begin{equation}
\vec{X}=\int d^dx \,\vec{x}\:\psi(\vec{x})^{\dag}\psi(\vec{x}) , \,
\end{equation} 
y utilizando (\ref{eq:conmut_psi_psid}) puede verse que $\vec{X}|\vec{x}\rangle=\vec{x}|\vec{x}\rangle$. Éstas partículas se encuentran localizadas en el espacio y al igual que $|\vec{p}\rangle$, satisfacen la relación de completitud para estados de una sola partícula
\begin{equation}
(\textbf{1})_1=\int d^dx\,|\vec{x}\rangle \langle\vec{x}| \,=\int \frac{d^dp}{(2\pi)^d}\, |\vec{p}\rangle \langle \vec{p}|,\,
\end{equation}
y están normalizados de forma que
\begin{equation}
\langle \vec{x}|\vec{x}\:'\rangle=\delta^{(d)}(\vec{x}-\vec{x}\:')\:\:\:\:\:\: \text{y} \:\:\:\:\:\:\langle \vec{p}|\vec{p}\:'\rangle=(2\pi)^d\delta^{(d)}(\vec{p}-\vec{p}\:').
\label{eq:norm_x}
\end{equation}
Además, la interpretación acerca de la localización es consistente con el cálculo
\begin{equation}
\langle \vec{x}|\vec{p} \rangle=e^{i\vec{p}\:\vec{x}},
\end{equation}
cuyo resultado se encuentra en el marco de la mecánica cuántica no relativista al utilizar la representación posición/momento de la función de onda.\\

El esquema de la existencia de estados localizados en posición se desvance en la teoría relativista. Si se hubiese seguido el mismo procedimiento de cuantización canónica, las soluciones para el campo de Klein Gordon (\ref{eq:action_KG}) $\phi_{KG}(\vec{x},t)$ hubiesen resultado\cite{Greiner:1996}:
\begin{equation}
\phi_{KG} (\vec{x},t)=\int \frac{d^dp}{(2\pi)^d}\,\frac{1}{\sqrt{2\tilde{\omega}_{\vec{p}}}} \left(\tilde{a}_{\vec{p}}\:e^{i(\vec{p}\cdot\vec{x}-\tilde{\omega}_{\vec{p}}\:t)}+\tilde{a}_{\vec{p}}^{\dag}\:e^{-i(\vec{p}\cdot\vec{x}-\tilde{\omega}_{\vec{p}}\:t)}\right)=\phi^{+}_{KG} (\vec{x},t)+\phi^{-}_{KG} (\vec{x},t),\,
\end{equation}
donde $\tilde{\omega}_{\vec{p}}\equiv \sqrt{\vec{p}^2+m^2}$. El factor de normalización que depende de la energía es tal que la medida de integración en el espacio de momentos $\int \frac{d^dp}{(2\pi)^d}\,\frac{1}{2\tilde{\omega}_{\vec{p}}}$ sea Lorentz-invariante \cite{Tong:2007}. Entonces, podríamos proceder de manera similar al caso no realtivista y definir\footnote{$\phi^+_{KG}$ indica la parte de $\phi_{KG}$ que contiene operadores de creación. También se podría haber definido $|\vec{x}\rangle_{KG}$ actuando con $(\phi^{\dag})^{+}_{KG}$ sobre $|0\rangle$.}
\begin{equation}
|\vec{x} \rangle_{\text{KG}}\equiv \phi^{-}_{KG}(\vec{x},0)|0\rangle=\int \frac{d^dp}{(2\pi)^d} \frac{1}{\sqrt{2\tilde{\omega}_{\vec{p}}}}\,e^{-i\vec{p}\:\vec{x}}|\vec{p}\rangle_{KG} ,\,
\end{equation}
donde $|\vec{p}\rangle_{KG} = \tilde{a}_{\vec{p}}^{\dag}|0\rangle$. Notamos que $\tilde{\omega}_{\vec{p}}\approx \sqrt{2m}$ cuando $|\vec{p}| \ll m$, con lo cual en dicho límite se recupera la expresión (\ref{eq:local_nr}) absorbiendo la constante en una redefinición de los operadores de creación/destrucción. Sin embargo, ocurre que la normalización de $|\vec{x}\rangle_{KG}$ no coincide con (\ref{eq:norm_x}). Más bien puede demostrarse que
\begin{equation}
\begin{split}
\langle \vec{x}|\vec{y} \rangle_{KG}=\int \frac{d^dp}{(2\pi)^d} \frac{e^{i\vec{p}(\vec{y}-\vec{y})}}{2\sqrt{p^2+m^2}}\,&\stackrel{d=3}{=}\frac{m}{2\pi^2 |\vec{x}-\vec{y}|}\int_0^{\infty} dt \sinh(t)\sin(m|\vec{x}-\vec{y}|\sinh(t))\\
&= \frac{mK_1(m|\vec{x}-\vec{y}|)}{2\pi^2|\vec{x}-\vec{y}|},
\end{split}
\end{equation}
siendo $K_1$ la función modificada de Bessel de segunda especie y de primer orden\footnote{Puede consultarse la \textit{Digital Library of Mathematical Functions} del NIST. Su página web es: \url{https://dlmf.nist.gov/} .} (o función de MacDonald), que tiene la particularidad de decaer exponencialmente con la distancia, es decir:
\begin{equation}
\langle \vec{x}|\vec{y} \rangle \longrightarrow \left(\frac{m^3}{2\pi|\vec{x}-\vec{y}|}\right)^{1/2} e^{-m|\vec{x}-\vec{y}|}\:\:\: \:\: \text{si}\:\:\: m|\vec{x}-\vec{y}| \gg 1.
\end{equation}
En vez de anularse para $\vec{x} \neq \vec{y}$ como en el caso no relativista, decae exponencialmente con la inversa de la longitud de onda de Compton $\frac{1}{m}$, o en unidades no naturales $\lambda_c=\frac{h}{mc}\approx 0,024 \: \textup{\r{A}}$. Por tanto, la teoría de Klein-Gordon introduce una escala natural a partir de la cual es factible hablar de la noción de localización de las partículas.
\section{Estados de muchas partículas}
Hasta el momento sólo se definieron estados $|\vec{p}\rangle$ y $|\vec{x}\rangle$ que fueron interpretados como estados de una sola partícula localizada en $\vec{p}$ y en $\vec{x}$ respectivamente. De manera muy natural, podría extenderse la definición para un número más grande de partículas. Por ejemplo, en el espacio de momentos se define   el vector $|\vec{p}_1\ldots \vec{p}_n\rangle$ tal que
\begin{equation}
|\vec{p}_1\ldots \vec{p}_n\rangle \propto a^{\dag}_{\vec{p}_{1}}
\ldots a^{\dag}_{\vec{p}_n}|0\rangle.
\end{equation}
La proporcionalidad se debe a que al haber cuantizado con reglas de conmutación, la estadística para un sistema de muchas partículas resulta bosónica, por lo cual, prodría ocurrir que hubiese más de una partícula con el mismo momento. Por otro lado, también se extiende la noción de un estado de muchas partículas localizadas en el espacio como
\begin{equation}
|\vec{x}_1 \ldots \vec{x}_n \rangle =\int \frac{d^dp_1}{(2\pi)^d}\ldots \frac{d^dp_n}{(2\pi)^d}\: e^{-i( \vec{p}_1 \:\vec{x}_1+\ldots+\vec{p}_n \:\vec{x}_n)} |\vec{p}_1\ldots \vec{p}_n\rangle .
\end{equation}
La existencia de una base de estados $|\vec{x}_1 \ldots \vec{x}_n \rangle$ en la teoría no relativista será una pieza clave para justificar más adelante que la entropía de entrelazamiento es nula en dicho límite.

\section{Cálculo del propagador y microcausalidad}
Un objeto central en cualquier teoría de campos es el propagador (o función de correlación de dos puntos o función de Green). Desde el punto de vista de la teoría axiomática de Wightman \cite{Wightman:2016}, el conocimiento de los correladores resultaría suficiente para describir completamente a la teoría. En las teorías libres, apelando al teorema de Wick\cite{Gaudin:1960}, puede demostrarse que cualquier función de correlación de $2n$ campos\footnote{En el caso del campo escalar, tiene que haber $n$ operadores de creación y $n$ operadores de destrucción para que el correlador no sea idénticamente nulo.} puede calcularse en términos de la función de correlación de dos puntos. En tiempo real, existen distintas prescripciones, todas extensiones analíticas del propagador en tiempo euclídeo. Sin embargo, sólo calcularemos el propagador a tiempo real conocido como propagador de Feynman\footnote{Como se verá en (\ref{eq:prop_feyn}) el mismo coincide con el propagador retardado en la teoría no relativista.} $G_F$, que se define como
\begin{equation}
\begin{split}
G_F(\vec{x}\:',t\:',\vec{x},t)&=\langle T \{\psi(\vec{x}\:',t\:')\psi^{\dag}(\vec{x},t)\}\rangle\\
&=\Theta (t\:'-t)\langle \psi(\vec{x}\:',t\:')\psi^{\dag}(\vec{x},t)\rangle + \Theta (t-t\:')\langle \psi^{\dag}(\vec{x},t)\psi(\vec{x}\:',t\:')\rangle \\
&=\Theta (t\:'-t)\langle \psi(\vec{x}\:',t\:')\psi^{\dag}(\vec{x},t)\rangle \\
&=G_R(\vec{x}\:',t\:',\vec{x},t).
\end{split}
\label{eq:prop_feyn}
\end{equation}
En la última igualdad se utilizaron las soluciones para $\psi$ y $\psi^{\dag}$ en términos de $a_{\vec{p}}$ y $a^{\dag}_{\vec{p}}$, y el hecho de que $\langle 0| a_{\vec{p}}a^{\dag}_{\vec{p}\:'}|0\rangle=\delta_{\vec{p},\vec{p}\:'}$ para demostrar la nulidad del último término. Explícitamente, el propagador se obtiene de la resolución de la integral:
\begin{equation}
G_F=\Theta (t\:'-t)\int \frac{d^dp}{(2\pi)^d}\,e^{i[\vec{p}\:(\vec{x}\:'-\vec{x})+\frac{\vec{p}^2}{2m}(t-t\:')]}.\,
\end{equation}
Completando cuadrados y sacando fuera de la integral el término independiente de $\vec{p}$,
\begin{equation}
G_F=\Theta (t\:'-t)e^{im\frac{(\vec{x}\:'-\vec{x})^2}{(t\:'-t)}}\int \frac{d^dp}{(2\pi)^d}\,e^{i\frac{(t-t\:')}{2m}[\vec{p}+m\frac{(\vec{x}\:'-\vec{x})}{t-t\:'}]^2}.\,
\end{equation}
Haciendo un cambio de variables por $\vec{y}:=\vec{p}+m\frac{(\vec{x}\:'-\vec{x})}{t-t\:'}$ ,
\begin{equation}
G_F=\Theta (t\:'-t)e^{im\frac{(\vec{x}\:'-\vec{x})^2}{2(t\:'-t)}}\int \frac{d^dy}{(2\pi)^d}\,e^{i\frac{(t-t\:')}{2m}\vec{y}^2}\,=\Theta (t\:'-t)e^{im\frac{(\vec{x}\:'-\vec{x})^2}{2(t\:'-t)}}\left(\frac{1}{2\pi}\int dy\, e^{-i\frac{(t\:'-t)}{2m}y^2} \, \right)^d.
\end{equation}
Utilizando la integral de Fresnel para $\alpha>0$, es decir, para $t\:'>t$.
\begin{equation}
\int dy\, e^{-i\alpha y^2} \,=\sqrt{\frac{\pi}{2\alpha}}(1-i),
\end{equation}
el propagador de Feynman en el límite no relativista resulta
\begin{equation}
G_F(\vec{x}\:',t\:',\vec{x},t)=\Theta (t\:'-t)\left(\frac{m}{2\pi i (t\:'-t)}\right)^{d/2}e^{i\frac{m(\vec{x}\:'-\vec{x})^2}{2(t\:'-t)}}.
\end{equation}
Para $t\rightarrow t\:'$ el límite puede calcularse introduciendo un camio de variable $\tau=i(t\:'-t)$
\begin{equation}
\lim_{\tau \rightarrow 0^{+}}\:G_F=\prod_{j=1}^{d}\left(\sqrt{\frac{m}{2\pi \tau}}e^{-\frac{m(x_j'-x_j)^2}{2\tau}}\right)=\delta^{(d)}(\vec{x}\:'-\vec{x}),
\end{equation}
resultado que concuerda con el valor de expectación en el vacío de (\ref{eq:conmut_psi_psid}), es decir, con $\langle 0|[\psi(\vec{x}\:',t\:'),\psi^{\dag}(\vec{x},t)]|0 \rangle=\delta^{(d)}(\vec{x}\:'-\vec{x})$.\\

El principio de microcausalidad establece que dos eventos con separación de tipo espacio, es decir, que satisfacen $(x-x\:')^2<0$, no ejercen ninguna influencia el uno sobre el otro (siendo $x=(\vec{x},t)$ y $x\:'=(\vec{x}\:',t\:')$). En general, cualquier observable de la teoría se puede escribir como \cite{Greiner:1996}
\begin{equation}
\hat{O}(x)=\hat{\psi}^{\dag}(x)O(x)\hat{\psi}(x),
\end{equation}
donde $O(x)$ es una función compleja o un operador diferencial. Por otro lado, puede demostrarse que
\begin{equation}
[\hat{O}(x),\hat{O}(x\:')]=O(x)O(x\:')\left(\hat{\psi}^{\dag}(x)\hat{\psi}(x\:')+\hat{\psi}^{\dag}(x\:')\hat{\psi}(x)\right)\Delta(x-x\:'),
\end{equation}
siendo $\Delta$ la función de Pauli-Jordan
\begin{equation}
\Delta(x\:'-x)=[ \psi(x\:'),\psi^{\dag}(x)].
\end{equation}
La propiedad de microcausalidad formalmente se enuncia\footnote{En \cite{Haag:1962} la conmutatividad está manifestada expresamente en el Postulado 7. Sin embargo, como en la teoría no relativista es imposible satisfacerla, los autores introducen el Postulado 8 que intuitivamente establece la existencia de una ecuación de campo que determina su evolución a todo tiempo dado el campo a un tiempo inicial.} como una relación de conmutación para eventos con separación de tipo espacio.
\begin{equation}
[\hat{O}(x),\hat{O}(x\:')]=0\:\:\:\:\:\:\text{si}\:\:\:\:\:\:(x-x\:')^2<0,
\end{equation}
y en términos de la función de Pauli-Jordan
\begin{equation}
\Delta(x-x\:')=0\:\:\:\:\:\:\text{si}\:\:\:\:\:\:(x-x\:')^2<0.
\end{equation}
En la teoría no relativista, también puede demostrarse que
\begin{equation}
G_F(\vec{x}\:',t\:',\vec{x},t)=\Theta (t\:'-t)\langle \Delta(x\:'-x)\rangle.
\label{eq:feyn_delta}
\end{equation}
Como $G_F\neq 0$ para $t\neq t'$ el principio de microcausalidad se desvanece pues la máxima velocidad de propagación es infinita $c\rightarrow \infty$ y los únicos eventos que no está influenciados causalmente son aquellos que posean el mismo $t$. Por otro lado, la expresión (\ref{eq:feyn_delta}) sólo es válidad en el límite no relativista. Para el campo de Klein-Gordon ocurre que si bien $G_F^{KG}$ nunca se anula, la función $\Delta$ sí lo hace para eventos de tipo espacio, respetando así al principio de microcausalidad.
%%% Local Variables:
%%% mode: latex
%%% TeX-master: "template"
%%% End:

%\chapter{Teorema de Reeh-Schlieder y entropía de entrelazamiento}\label{cap3}
\chapterquote{Admiro la elegancia de su método de cálculo, debe ser muy hermoso cabalgar por esos campos sobre el caballo de las verdaderas matemáticas, mientras que los que son como nosotros tenemos que hacer nuestro camino a pie.}{Albert Einstein en referencia a Tullio Levi-Civita.}

En el marco de la teoría axiomática de campos, usualmente conocida en inglés como \textit{Algebraic Quantum Field Theory}(AQFT) o \textit{Local Quantum Physics} \cite{Haag:1992hx}, como consecuencia inesperada de los axiomas Wightman\cite{Wightman:2016}, H. Reeh y S. Schlieder probaron un teorema que causó revuelo y cuyo resultado se encuentra fuertemente ligado con cuestiones de localización y entrelazamiento \cite{reeh_sch}. Por ejemplo, en una prueba reciente del teorema-a \cite{Casini:2017vbe} que utiliza estados Markovianos (aquellos que saturan a la subaditividad fuerte), como consecuencia del teorema de Reeh-Schlieder, deben considerarse regiones limitadas sólo en el plano nulo del cono de luz.

En términos formales, el teorema predice la verificación de una propiedad por parte del vacío denominada \textit{ciclicidad} y en conjunto con la propiedad de microcausalidad de la teoría relativista implican su \textit{separabilidad}. Consecuentemente, éste posee correlaciones entre cualesquiera dos regiones y por ende, la entropía de entrelazamiento de una región arbitaria resulta no nula. Dentro de este contexto, el objetivo de este capítulo consiste en entender como en el límite no relativista en donde la máxima velocidad de propagación es infinita, dos aspectos a priori contradictorios tienen lugar: la validez del teorema de Reeh-Schlieder, y la nulidad de la entropía de entrelazamiento. 



\section{Enunciado}
El objetivo de esta sección es enunciar el teorema de Reeh-Schlieder siguiendo los pasos de \cite{Witten:2018lha}. Para empezar, consideremos algunas definiciones:

\begin{itemize}
\item $M_D$ el espacio-tiempo de Minkowski de dimensión $D$, es decir $D=d+1$, con signatura $(-+\ldots+)$\footnote{Sólo en este capítulo se considerará esta signatura para la métrica.}.
\item $\Sigma\subset M_D$ una hipersuperficie completa de tipo espacio (o hipersuperficie de Cauchy). Clásicamente es una región del espacio tiempo en donde se determinan las condiciones inciales de la teoría.
\item $\mathcal{V}\subset \Sigma$ un conjunto abierto arbitrario de la hipersuperficie.
\item $\mathcal{U}_{\mathcal{V}} \subset M_D$ un pequeño entorno de $\mathcal{V}$ en el espacio-tiempo.
\end{itemize}
Para visualizar los objetos recien definidos podríamos pensar en $D=2$, tomando $\Sigma=\{x\in M_D:\: x^0=0\}$, $\mathcal{V}=\{x\in M_D:\: x^0=0\:\wedge\:|x^1|<L\}$ y $\mathcal{U}_{\mathcal{V}}=\{x\in M_D:\: |x^0|<\epsilon\:\wedge\:|x^1|<L\}$.

\begin{figure}[ht]
    \centering
    \includegraphics[width=0.4\linewidth]{Reeh_sch_intro.png}
    \caption{Ejemplo de los conjuntos $\Sigma$, $\mathcal{V}$ y $\mathcal{U}_{\mathcal{V}}$ en $D=2$.}
    \label{fig:examp}
 \end{figure}

Además, consideremos:
\begin{itemize}
\item $|\Omega\rangle$ el estado de vacío, es decir, el estado de mínima energía.
\item Si $\phi(x^{\mu})$ representa un campo escalar, los valores de expectación del tipo $\langle \Omega |\phi(x^{\mu}_1)\ldots\phi(x^{\mu}_n)|\Omega \rangle$  no se encuentran bien definidos (poseen divergencias). Por tal motivo, se considerará la funcional $\phi_f$ del campo $\phi$ en términos de una función suave $f$ como: $\phi_f=\int d^Dx\:f(x^u)\phi(x^{\mu})$.
\item $|\psi_{\vec{f}}\rangle = \phi_{f_1}\phi_{f_2}\ldots\phi_{f_n}|\Omega\rangle$ el estado en el espacio de Hilbert $\mathcal{H}$ con las $f_i$ con soporte en $\mathcal{U}_{\Sigma}$.
\item $\mathcal{H}_0\subset\mathcal{H}$ el sector de vacío consistente de todos los estados que resultan ser combinación lineal de $|\psi_{\vec{f}}\rangle$, en otras palabras, $\mathcal{H}_0=\text{Span}\{|\psi_{\vec{f}}\rangle\}|_{f_{i\in \mathcal{U}_{\Sigma}}}$.
\end{itemize}

\textbf{Teorema de Reeh-Schlieder}
Considerando todas las definiciones, si las $f_i$ están restringidas en un entorno arbitrariamente chico $\mathcal{U}_{\mathcal{V}}\subset M_D$ entonces
$\mathcal{H}_0=\text{Span}\{|\psi_{\vec{f}}\rangle\}|_{f_{i\in \mathcal{U}_{\mathcal{V}}}}$.\\

En el marco de la Teoría Axiomática de Campos, resulta más riguroso hablar del álgebra de operadores en una región que de los operadores de campos\textit{ per se}. Esto da una formulación de la teoría que no depende particularmente de los campos utilizados en su descripción\cite{Casini:2019qst}. Por ese motivo, introducimos dos nuevas definiciones:
\begin{itemize}
\item $\mathcal{V}\subset \Sigma$ un conjunto abierto, entonces $\mathcal{A}_{\mathcal{U}_{\mathcal{V}}}$ se define como el álgebra de operadores con soporte en $\mathcal{U}_{\mathcal{V}}$.
\item Para $|\psi\rangle \in \mathcal{H}_0$ y $\mathcal{A}_{\mathcal{U}_{\mathcal{V}}}$ un álgebra de operadores, se dice que $\psi$ es un \textit{vector cíclico} de $\mathcal{A}_{\mathcal{U}_{\mathcal{V}}}$ si $\mathcal{H}_0=\text{Span}\{a|\psi\rangle:\:a\in \mathcal{A}_{{\mathcal{U}_{\mathcal{V}}}}\}$.\\
\end{itemize}
\textbf{Teorema de Reeh-Schlieder (nuevo enunciado)}  El vacío $|\Omega\rangle$ es un vector cíclico para $\mathcal{A}_{\mathcal{U}_{\mathcal{V}}}$, con $\mathcal{V}\subset \Sigma$ arbitrariamente pequeño.  
 
\section{Demostración}
\subsection{Caso relativista}
Para probar el teorema utilizaremos el método de \textit{reductio ad absurdum}. Supongamos que el teorema de Reeh-Schlieder es falso, entonces $\mathcal{H}_0\neq\text{Span}\{|\psi_{\vec{f}}\rangle\}|_{f_{i\in \mathcal{U}_{\mathcal{V}}}}$. Luego, un vector $|\chi\rangle$ distinto de $|\Omega\rangle$ debe existir tal que\footnote{Estamos pensando que para un conjunto abierto arbitrario $U$, $\mathcal{H}_0=U\oplus U^{\perp}$ por ser $\mathcal{H}_0$ separable.} $|\chi\rangle\in\left(\text{Span}\{|\psi_{\vec{f}}\rangle\}|_{f_{i\in \mathcal{U}_{\mathcal{V}}}}\right)^{\perp}$, en otras palabras
\begin{equation}
0=\langle \chi|\psi_{\vec{f}}\rangle,
\end{equation}
con $f_i$ con soporte en $\mathcal{U}_{\mathcal{V}}$. Esta afirmación es equivalente aún si los campos no son considerados como distribuciones:
\begin{equation}
0=\langle \chi|\psi_{\vec{f}}\rangle \:\:\:f_i\text{ con soporte en }\mathcal{U}_{\mathcal{V}} \:\:\: \Leftrightarrow \:\:\:\langle \chi|\phi(x_1)\phi(x_2)\ldots\phi(x_n)|\Omega\rangle=0\:\:\:\:x_i \in \mathcal{U}_{\mathcal{V}}.
\label{eq:cero_reeh}
\end{equation}
\textbf{Lema} Si (\ref{eq:cero_reeh}) es válida $\forall \:x_i \in \mathcal{U}_{\mathcal{V}}$ $\Rightarrow$ es también válida $\forall \:x_i \in M_D$.\\

Si el lema es verdadero entonces $|\chi\rangle$ es ortogonal a todo vector en $\mathcal{H}_0$\footnote{Notemos que eso es equivalente a decir que $|\chi\rangle \in \mathcal{H}^{\perp}_0$. Pero por otro lado, $\mathcal{H}_0=\mathcal{H}_0\oplus \{|\Omega\rangle \}$ con lo cual $\mathcal{H}^{\perp}_0=\{|\Omega\rangle\}$ y $|\chi\rangle =|\Omega\rangle$.} (ya que $\mathcal{H}_0=\text{Span}\{|\psi_{\vec{f}}\rangle\}|_{f_{i\in \mathcal{U}_{\Sigma}}}$ debido a la definición del sector de vacío) implicando que $|\chi\rangle=|\Omega\rangle$, lo cual es un absurdo. Por lo tanto, el teorema de Reeh-Schlieder queda demostrado.\\
Ahora la tarea difícil, la \textbf{demostración del lema}:
\begin{itemize}
\item Definamos $\varphi(x_1,\ldots,x_n)\equiv\langle\chi|\phi(x_1)\ldots \phi(x_n)|\Omega\rangle$. Primero, demostraremos que $\varphi$ continua anulandose si $x_n$ se mueve por fuera de $\mathcal{U}_{\mathcal{V}}$, manteniendo las otras variables en $\mathcal{U}_{\mathcal{V}}$. Segundo, probaremos que sin restricciones en $x_{n-1}$ y en $x_n$, manteniendo $x_1,\ldots,x_{n-2}\in \mathcal{U}_{\mathcal{V}}$, $\varphi$ continua anulandose. Finalmente, procediendo de forma similar variando la coordenada $k$-ésima hasta que $k=n$, concluimos que $\varphi$ es idénticamente nula $\forall\:x_i\in M_D$, demostrando así el lema.  

\item Para demostrar que $\varphi$ se anula cuando sólo se varía $x_n$, consideremos un vector de tipo tiempo que apunta en la dirección futura $R\in M_D$ y un parámetro real $u$, tales que $x_n'=x_n+uR$ (ver (\ref{fig:Raxis})(a)). Si definimos $g$ como
$\varphi$ evaluada en $x_n'$:
\begin{equation}
g(u)=\langle\chi|\phi(x_1)\ldots(x_n+uR)|\Omega\rangle=\langle\chi|\phi(x_1)\ldots e^{i(uR^0H-u\vec{R}\vec{p})}\phi(x_n)e^{-i(uR^0H-u\vec{R}\vec{p})}|\Omega\rangle.
\end{equation}
Dado que $P^u|\Omega\rangle=0$,
\begin{equation}
g(u)=\langle\chi|\phi(x_1)\ldots e^{i(uR^0H-u\vec{R}\vec{p})}\phi(x_n)|\Omega\rangle.
\end{equation}
\begin{figure}[ht]
    \centering
    \includegraphics[width=0.6\linewidth]{Re_timelike.png}
    \caption{En concordancia con el ejemplo de (\ref{fig:examp}), en (a) $R\in M_D$ es un vector de tipo tiempo que apunta en la dirección futura que define un rayo $uR$ a través del origen. La función $g(u)$ se anula en el segmento amarillo, ó como está indicado en (b) en el intervalo $I$ de $\text{Re}(u)$. Además, para demostrar el teorema $u\in \mathbb{R}\rightarrow u \in \mathbb{C}$.}
    \label{fig:Raxis}
 \end{figure}
Es importante remarcar que si $u$ es tal que $x_n+uR\in \mathcal{U}_{\mathcal{V}}$, eso implica que $g(u)$ se anula para los correspondientes valores de $u$ en $I$\footnote{El intervalo $I=(-\epsilon,\epsilon)$ consiste de todos los valores de $u$ sobre el rayo que genera $R$ dentro de $\mathcal{U}_{\mathcal{V}}$, tales que $g(u)=0$.}. De ahora en adelante $u\in \mathbb{C}$. Afirmamos que como $|\vec{R}|<R^0$ por ser $R$ de tipo tiempo, entonces $\text{Im}(u|\vec{R}|)<\text{Im}(uR^0)$ para $u$ en el semiplano superior $\mathbb{H}$. Más aún, usando el hecho de que $\vec{R}\vec{p}\leq |\vec{R}||\vec{p}|$ y que $|\vec{p}|\leq H$ (porque $H=+\sqrt{p^2+m^2}$) puede notarse que
\begin{equation}
\text{Im}(u|\vec{R}|)<\text{Im}(uR^0)\:\:\: u\in \mathbb{H} \:\:\:\:\Rightarrow \:\:\:\: \left|e^{i(uR^0H-u\vec{R}\vec{p})}\right|<1\:\:\: u\in \mathbb{H}.
\label{eq:bound}
\end{equation}
Entonces, el operador exponencial se encuentra acotado para los $u$ en el semiplano superior y consecuentemente, resulta holomorfo en la misma región.
\item Hemos aprendido que $g(u)$ es holomorfa en el semiplano superior, continua a medida que $u$ se acerca al eje real por encima, y nula en el segmento $I=(-\epsilon,\epsilon)$ del eje real. Consecuentemente (ver Apéndice \ref{app:schwarz}) $g(u)=0\:\:\forall u\in \mathbb{C}$. En particular, para $u\in \mathbb{R}$ concluimos que $g(u)$ se anula para un rayo en $M_D$ que pasa a través de $x_n$. Continuando con el mismo procedimiento podemos probar que $\varphi$ se anula en el cono de luz pasado y futuro de $x_n$. Es posible repetir el mismo argumento pero eligiendo otro conjunto de $M_D$ en donde $\varphi$ se anule, y comenzar a zig-zagear en $M_D$ hasta cubrirlo enteramente. Como consecuencia, conluimos que $\varphi=0\:\:\forall x_1,\ldots x_{n-1}\in \mathcal{U}_{\mathcal{V}}$, sin la restricción de que $x_n\in \mathcal{U}_{\mathcal{V}}$.
\item El próximo paso consiste en remover la restricción $x_{n-1}\in \mathcal{U}_{\mathcal{V}}$. Sólo moveremos las últimas dos variable de forma similar al paso anterior $x_{n-1},x_n\rightarrow x_{n-1}+uR,x_n+uR$. Luego, puede verse que
\begin{equation}
g(u)=\langle \chi|\phi(x_1)\ldots e^{i(uR^0H-u\vec{R}\vec{p})} \phi(x_{n-1})\phi(x_n)|\Omega \rangle.
\end{equation}
De nuevo, concluimos que $g(u)$ se anula en $u\in \mathbb{C}$ y por lo tanto $\varphi=0$ para un desplazamiento arbitrario en $M_D$ de ambos $x_{n-1},x_n$ al mismo tiempo. Dado que somos libres de mover independientemente a $x_n$, concluimos que $\varphi=0\:\:\forall x_1,\ldots x_{n-2}\in \mathcal{U}_{\mathcal{V}}$, sin la restricción $x_{n-1},x_n\in \mathcal{U}_{\mathcal{V}}$. Finalmente, como se mencionó al principio de la demostración del lema, realizamos el mismo procedimiento para las últimas $k$ coordenadas hasta que $k=n$, lo cual completa la demostración.
\end{itemize}
$\:\:\:\:\:\:\:\:\:\:\:\square$
\subsection{Caso no relativista}
Para extender la demostración al caso no relativista el hamiltoniano $H$ de la teoría debe ser tal de que exista un estado de mínima energía, es decir, un estado de vacío $|\Omega\rangle$. También, debe pedirse que $|\vec{p}|\leq H$ para poder probar la implicación (\ref{eq:bound}). Notemos que esto último trivialmente se satisface en la teoría relativista pues $H=\sqrt{p^2+m^2}>|\vec{p}|$. Sin embargo, si ingenuamente tratásemos de utilizar la relación no relativista usual $H=\frac{p^2}{2m}$, no podríamos concluir que dicha desigualdad se cumple. No obstante, considerando el término dominante $m$ de la relación de dispersión, es decir $H=m+\frac{p^2}{2m}$, es posible probar que $H\geq 2\sqrt{m\cdot \frac{p^2}{2m}}\geq |\vec{p}|$ (en donde hemos utilizado la desigualdad de las medias AM-GM). Notemos que una vez satisfecha la relación $H\geq |\vec{p}|$, la demostración no requiere del uso de la invarianza de Lorentz, más bien de la existencia de traslaciones espacio-temporales, que particularmente se encuentran bien definidas aun en el límite no relativista. Con esto último, hemos extendido la prueba para el caso no relativista y por ende probado la ciclidad del vacío para esta teoría.$\: \square$

Una cuestión a tener en cuenta es que las superficies de Cauchy necesariamente son hipersupeficies con $t$ constante pues en el caso no relativista $c\rightarrow \infty$ y el cono de luz colapsa a una región que abarca todo $M_D$ menos la recta $x^0=0$. Por otro lado, también notamos que la ciclidad del vacío en la teoría no relativista no es posible obtenerla naturalmente sin haber partido \textit{ab initio} de una teoría relativista.


\section{Corolarios}
Como mencionamos al principio del capítulo, la ciclidad del vacío en conjunto con la propiedad de microcausalidad de la teoría relativista implican la denominada separabilidad del vacío. En términos del álgebra de operadores se dice que un vector $|\psi\rangle \in \mathcal{H}_0$ es separador del álgebra de operadores $\mathcal{A}_{\mathcal{U}_{\mathcal{V}}}$ si la condición $a|\psi\rangle=0$ con $a\in \mathcal{A}_{\mathcal{U}_{\mathcal{V}}}$ implica $a=0$.\\

\textbf{Separabilidad del vacío} Si el vacío $|\Omega \rangle$ es cíclico para $\mathcal{A}_{\mathcal{U}_{\mathcal{V}}}$ (resultado que sigue del teorema de Reeh-Schlieder) y $\mathcal{A}_{\mathcal{U}_{\mathcal{V'}}}$ es un álgebra de operadores de una región $\mathcal{U}_{\mathcal{V'}}$ con una separación de tipo espacio respecto de $\mathcal{U}_{\mathcal{V}}$\footnote{Notemos que $\mathcal{U}_{\mathcal{V'}}$ no es el álgebra de operadores conmutante con $\mathcal{U}_{\mathcal{V}}$, denotada como $\mathcal{U}_{\mathcal{V}}'$. En realidad se cumple que $\mathcal{U}_{\mathcal{V'}}\subseteq  \mathcal{U}_{\mathcal{V}}'$, y cuando coinciden dicha condición se denomina dualidad de Haag \cite{Witten:2018lha}. Como la demostración solo usa la conmutatividad de los operadores, la separabilidad del vacío sigue siendo válida aún considerando $\mathcal{U}_{\mathcal{V}}'$. La pérdida de la dualidad está relacionada con la existencia de sectores de superselección\cite{Casini:2019kex}.}, entonces $|\Omega \rangle$ es un vector separador de $\mathcal{A}_{\mathcal{U}_{\mathcal{V'}}}$.\\

Para realizar la demostración consideremos dos regiones con separación de tipo espacio $\mathcal{U}_{\mathcal{V}}$ y $\mathcal{U}_{\mathcal{V'}}$, y un operador $a$ con soporte en $\mathcal{U}_{\mathcal{V}}$. Por la propiedad de microcausalidad se cumple que:
\begin{equation}
[\phi(x),a]=0\:\:\:\:\: x \in \mathcal{U}_{\mathcal{V'}}.
\end{equation}
Análogamente, considerando un operador $a'$ con soporte en $\mathcal{U}_{\mathcal{V'}}$
\begin{equation}
[\phi(x),a']=0\:\:\:\:\: x \in \mathcal{U}_{\mathcal{V}}.
\end{equation}
Supongamos además que $a$ es un operador de destrucción del vacío $a|\Omega \rangle=0$. La conmutatividad de $a$ con operadores $\phi(x_i)$ con soporte en $x_i\in \mathcal{U}_{\mathcal{V'}}$ implica que:
\begin{equation}
a \phi(x_1) \ldots \phi(x_n)|\Omega \rangle=0\:\:\:\: x_i \in \mathcal{U}_{\mathcal{V'}}.
\end{equation}
Pero debido al teorema de Reeh-Schlieder, $\phi(x_1) \ldots \phi(x_n)|\Omega \rangle$ genera todo $\mathcal{H}_0$. Por lo tanto $a=0$ en todo el sector de vacío $\mathcal{H}_0$. $\square$\\

Este resultado de la teoría relativista es sumamente interesante y como ya se mencionó anteriormente, depende fuertemente de la conmutatividad de los operadores (es decir, de la propiedad de microcausalidad de la teoría). Existen otro resultados que también pueden desprenderse de la separabilidad del vacío \cite{Witten:2018lha}: la existencia de estados de energía negativa y de correlaciones en el vacío entre regiones con separación de tipo espacio.

Aunque el teorema de Reeh-Schlieder es válido en la teoría no relativista, los corolarios no siguen siendo necesariamente ciertos pues se pierde la conmutatividad a tiempos distintos. En particular, puede probarse que la entropía de entrelazamiento de una región del vacío con su complemento es siempre nula \cite{Hason:2017flq}. En lo siguiente, introduciremos brevemente la noción de entropía de entrelazamiento y se darán argumentos por la cual la misma es nula en la teoría no relativista.


\section{Entropía de entrelazamiento}
Al estudiar un sistema físico con propiedades inherentes de la mecánica cuántica, resulta imposible separar la noción del “estado actual del sistema” respecto de “nuestro conocimiento del estado del sistema” \cite{Headrick:2019eth}. Ambas pretenciones se encuentran codificadas en un operador $\rho$ que se denomina matriz densidad y posee las siguientes propiedades:
\begin{equation}
\rho=\rho^{\dag},\:\:\:\:\:\: \text{Tr}(\rho)=1.
\end{equation}
En el formalismo de la matriz densidad, puede calcularse el  valor de exprectación de un observable arbitrario $\mathcal{O}$ como
\begin{equation}
\langle \mathcal{O} \rangle_{\rho} = \text{Tr}(\mathcal{O} \rho).
\label{eq:val_expect}
\end{equation}
Motivada por la definición de la entropía de Shannon en el contexto de la teoría de la información, se define la entropía de Von Neumann como
\begin{equation}
S(\rho)=-\text{Tr}(\rho \:\text{log}(\rho)).
\label{eq:EE_form}
\end{equation}
En el marco de Teoría Cuántica de Campos la entropía de Von Neumann tiene un rol protagónico cuando un observador sólo puede acceder a un subconjunto del conjunto completo de observables de la teoría \cite{Casini:2009sr}. Dada una región $V$ del espacio, para lidiar con la falta de acceso a los observables en su complemento $\bar{V}$, se define la matriz densidad reducida en $V$ como\footnote{Se considera como matriz densidad del sistema completo a $\rho=|0\rangle \langle 0|$.}
\begin{equation}
\rho_V=-\text{Tr}_{\bar{V}}(|0\rangle \langle 0|).
\label{eq:rho_red}
\end{equation} 
Consecuentemente, la \textbf{entropía de entrelazamiento} asociada a la región $V$ se define como la entropía de Von Neumann de $\rho_V$:
\begin{equation}
S(V)=-\text{Tr}(\rho_V \text{ log}(\rho_V)).
\label{ec:def_EE}
\end{equation}
\section{Entropía de entrelazamiento de la teoría no relativista}
Para entender por qué la entropía de entrelazamiento de una región $V$ arbitraria en la teoría no relativista es cero \cite{Hason:2017flq}, pensemos que el espacio de Hilbert $\mathcal{H}$\footnote{No siempre el espacio de Hilbert total $\mathcal{H}$ puede descomponerse de esa manera. Para un campo fermiónico libre si es posible hacerlo, mientras que para teorías de gauge no \cite{Casini:2013rba}.} es tal que
\begin{equation}
\mathcal{H}=\mathcal{H}_V \otimes \mathcal{H}_{\bar{V}}.
\end{equation}
Como se argumentó en el capítulo anterior, en la teoría no relativista existe una base de estados localizados, por ende, consideraremos una base de estados localizados en $V$ para $\mathcal{H}_V$ y una en $\bar{V}$ para $\mathcal{H}_{\bar{V}}$. Esto último resulta equivalente a decir que el operador número restringido a la región $V$, $N_V$, conmuta con el operado número restringido a la región $\bar{V}$, $N_{\bar{V}}$ respectivamente. Dado que en el vacío no hay partículas $0=N=N_V+N_{\bar{V}}$ y al ser $N_i$ semidefinidos positivos \footnote{En la teoría relativista, la carga conservada que resulta de la simetría $U(1)$ para el campo escalar es $N=N_{\text{partículas}}-N_{\text{antipartículas}}$. En tal caso, $N$ no posee la propiedad de ser semidefinido positivo.}, concluimos que $N_V=N_{\bar{V}}=0$. Esto último, implica la separabilidad del vacío global como el producto de vacíos locales en $V$ y en $\bar{V}$:
\begin{equation}
|0\rangle = |0\rangle _{V} \otimes |0\rangle_{\bar{V}}.
\end{equation}
Entonces, al trazar sobre $\bar{V}$ resulta que $\rho_V=|0 \rangle \langle 0|_V $ representa un estado puro y en virtud de (\ref{ec:def_EE}), $S(V)=0$.
%%% Local Variables:
%%% mode: latex
%%% TeX-master: "template"
%%% End:

%\chapter{Entropía de entrelazamiento a densidad finita}\label{cap4}
\chapterquote{Un segundo bastó para separar su cabeza del cuerpo, pasarán siglos para que una cabeza como aquella vuelva a ser llevada sobre los hombros de un hombre de ciencias}{J.L. Lagrange sobre la ejecución de A. Lavoiser (1794).}

La entropía de entrelazamiento es un objeto que permite medir las correlaciones de una región del espacio $V$ en una teoría, con su complemento $\bar{V}$. En el presente capítulo consideraremos un campo de Dirac libre discretizado en 1+1 dimensiones espacio-temporales y estudiaremos la entropía de entrelazamiento siguiendo los pasos de \cite{Casini:2009sr}. En particular, introduciremos una densidad no trivial de materia, es decir, en el marco del formalismo gran canónico, un potencial químico $\mu$ no nulo. Por el momento, se han hecho cálculos analíticos considerando la teoría en el discreto y se ha procedido a una etapa de programación para la obtención de las curvas de entropía de entrelazamiento. Queda aun el trabajo de que las curvas de $S(V)$ representen la entropía de entrelazamiento en el continuo de la teoría. 

\section{Densidad de materia no nula}
Consideremos un campo fermiónico libre $\psi$ de masa $m$ en $d+1$ dimensiones espacio-temporales. Su densidad lagrangiana resulta\footnote{Se utilizará la métrica Lorentziana $g_{\mu \nu}=(+-\ldots -)$ con $d+1$ coordenadas espacio-temporales.}
\begin{equation}
\mathcal{L}=\bar{\psi}(i\gamma^{\nu}\partial_{\nu}-m)\psi.
\label{eq:lag_libre}
\end{equation}
Incluir una densidad de materia no nula es equivalente a exigir que el valor de expectación de una carga conservada sea distinto del trivial \cite{Solis:2016}. Para el caso del campo fermiónico, la simetría global $U(1)$ dada por $\psi \mapsto e^{-i\alpha}\psi$ con $\alpha$ constante, por el teorema de Noether implica la existencia de una corriente conservada $j^{\mu}$ dada por:
\begin{equation}
j^{\mu}=\frac{\partial \mathcal{L}}{\partial(\partial_{\mu}\psi)}\frac{\delta \psi }{\delta \alpha}=\bar{\psi}\gamma ^{\mu}\psi.
\end{equation}
Donde la correspondiente carga conservada $N$ viene dada por:
\begin{equation}
N=\int d^dx\:j^0=\int d^dx \:\psi^{\dag}\psi.
\end{equation}
En este caso $N$ representa el número de partículas.\\

La descripción de un sistema físico que está en contacto con un reservorio que permite el intercambio de energía y de partículas en donde la densidad media de energía y de materia (densidad media de partículas) son constantes puede realizarse en el marco de la descripción gran canónica. En la misma, el reservorio posee una temperatura $T$ y un potencial químico $\mu$ dados. En este formalismo, la densidad hamiltoniana se ve modificada:
\begin{equation}
\mathcal{H}\mapsto \mathcal{H}'=\mathcal{H}-\mu N.
\end{equation}
Con lo cual la densidad lagrangiana (\ref{eq:lag_libre}) con un potencial químico arbitrario $\mu$ resulta
\begin{equation}
\mathcal{L}=\bar{\psi}(i\gamma^{\nu}\partial_{\nu}-m)\psi+\mu \psi^{\dag}\psi =\bar{\psi}\left((p_0+\mu)\gamma^0+p_i\gamma^i-m\right)\psi.
\label{eq:ec_dirac}
\end{equation}
Redefiniendo $\tilde{p_0}=p_0+\mu$, la relación de dispersión se modifica como
\begin{equation}
p_0=\tilde{p_0}-\mu=-\mu \pm \sqrt{p^2+m^2}.
\end{equation}
En función del signo de $p_0$ los estados en el espacio de momentos $\vec{p}$ quedan separados por la denominada \textit{superficie de Fermi} que consiste de aquellos estados con momento $\vec{p}$ tales que $p_0(\vec{p})=0$, es decir, aquellos estados con momento $|\vec{p}|=\sqrt{\mu^2-m^2}$.
\section{Entropía de entrelazamiento de un sistema fermiónico discretizado \label{sec:EE_lattice_f}} 
Los primeros cálculos de entropía de entrelazamiento se abordaron con una técnica que se denomina de \textit{tiempo real}\footnote{En contraste con la técnica que se denominda de \textit{tiempo euclídeo}.}, en la cual se parte de una versión discretizada de la teoría de campos y eventualmente se toma el límite al continuo \cite{Casini:2009sr}. El objetivo de esta técnica consiste en calcular la matriz densidad correspondiente al estado de vacío global utilizando correladores. Si $\rho_V$ resulta la matriz reducida (\ref{eq:rho_red}) de una región $V$, por cada operador $\mathcal{O}_V$ localizado dentro de $V$, los valores de expectación en el vacío $\langle \mathcal{O}_V \rangle$, de forma similar a  (\ref{eq:val_expect}), deben ser tales que \cite{Casini:2009sr}\cite{Casini:2005rm}
\begin{equation}
\langle \mathcal{O}_V \rangle = \text{Tr}(\rho_V\mathcal{O}_V).
\end{equation}
Al considerar una teoría discretrizada de fermiones libres, su densidad hamiltoniana $\mathcal{H}$ puede escribirse en términos de operadores de creación y destrucción localizados en cada sitio
\begin{equation}
\mathcal{H}=\sum_{i,j}H_{ij}c_i^{\dag}c_j,
\label{ec:ham_dic}
\end{equation}
tales que $\lbrace c_i,c_j^{\dag}\rbrace=\delta_{ij}$, al igual que en un modelo de tight-binding. Al ser una teoría libre (cuadrática o gaussiana), y en virtud del teorema de Wick \cite{peschel2002}, la matriz densidad reducida $\rho_V$ resulta
\begin{equation}
\rho_V=\frac{e^{-\mathcal{K}}}{\text{Tr}(e^{-\mathcal{K}})}=\frac{e^{-\sum_{i,j\in V}K_{ij}c_i^{\dag}c_j}}{\text{Tr}(e^{-\mathcal{K}})},
\end{equation}
donde $\mathcal{K}$ es el denominado \textit{hamiltoniano modular}. De particular interés resulta el correlador
\begin{equation}
C_{ij}=\langle c_i^{\dag}c_j \rangle ,
\end{equation}
ya que el hamiltoniano modular podrá escribirse en términos de $C$. Al realizar una transformación unitaria\footnote{Para mantener invariante las relaciones de anticonmutación, es decir, para que $\lbrace a_k,a_l^{\dag}\rbrace =\delta_{kl}$ .} tal que diagonalice al hamiltoniano modular
\begin{equation}
c_i=\sum_k \phi_k(i)a_k,
\end{equation}
siendo $\lbrace a_i \rbrace$ un nuevo conjunto de operadores. Puede demostrarse que
\begin{equation}
C_{ij}  =\sum_k \phi^{*}_k(i)\phi_k(j)\frac{1}{e^{ \epsilon_k}+1},
\label{eq:cor_pes}
\end{equation}
y que 
\begin{equation}
K_{ij}=\sum_k \phi^{*}_k(i)\phi_k(j)\epsilon_k,
\end{equation}
donde $\epsilon_k$ corresponde a la energía del $k$-ésimo estado del hamiltoniano modular. Estas dos últimas expresiones permiten escribir al hamiltoniano modular como
\begin{equation}
\mathcal{K}=-\log(C^{-1}+1).
\end{equation}
Entonces la matriz densidad reducida $\rho_V$ queda exclusivamente definida a través de la matriz de correlación de dos puntos $C$ restringida en $V$. Por último, la entropía de entrelazamiento $S(V)$ puede calcularse según (\ref{eq:EE_form}):
\begin{equation}
\begin{split}
S(V)=-\sum_k \text{Tr}(\rho_k \log(\rho_k)) & =\sum_k \left(\frac{\epsilon_k}{1+e^{ \epsilon_k}}+\log(1+e^{- \epsilon_k})\right) \\
& = -\text{Tr}\left(C\log(C)+(1-C)\log(1-C)\right)
 \\
& = \sum_j \left(\nu_j \log(\nu_j)+(1-\nu_j)\log(1-\nu_j) \right),
\end{split}
\label{eq:EE_matC}
\end{equation}
donde $\nu_j$ indican los autovalores de $C$ restringidos a una región $V$.\\

En conclusión, por tratarse de una teoría cuadrática (libre), tanto la matriz densidad reducida $\rho_V$ como la entropía de entrelazamiento $S(V)$ dependen solamente de la matriz de correlación de dos puntos restringida en $V$. Entonces, la tarea de computar $S(V)$ se reduce al cálculo de los autovalores de la función de correlación de dos puntos restringida a $V$.
\section{Entropía de entrelazamiento del campo de Dirac en 1+1}

Consideremos como nuestro sistema a un campo fermiónico libre $\psi$ en contacto con un baño térmico a temperatura $T=0$ y potencial químico $\mu$. En vez de considerar la densidad lagrangiana (\ref{eq:ec_dirac}), consideraremos la versión simetrizada que posee la particularidad de ser hermítica\cite{Greiner:1996}
\begin{equation}
\mathcal{L}=\mathcal{L}^{\dag}=\frac{i}{2}\bar{\psi}\gamma^{\nu}\stackrel{\leftrightarrow}{\partial_{\nu}}\psi-m\bar{\psi}\psi +\mu \psi^{\dag}\psi ,
\end{equation}
con lo cual
\begin{equation}
\mathcal{H}(\psi,\psi^{\dag})=-\frac{i}{2}\bar{\psi}\gamma^{j}\stackrel{\leftrightarrow}{\partial_{j}}\psi+m\bar{\psi}\psi -\mu \psi^{\dag}\psi ,
\end{equation}
donde el operador $\stackrel{\leftrightarrow}{\partial}$ es tal que $A\stackrel{\leftrightarrow}{\partial}B=A(\partial B) - B (\partial A)$. Trabajaremos en 1+1 dimensiones espacio-temporales, y discretizaremos la coordenada espacial con un espaciado $\epsilon$  de forma tal de escribir el hamiltoniano de la teoría como (\ref{ec:ham_dic}). Para pasar la teoría al discreto efectuamos:
\begin{equation}
\psi \mapsto \psi_n,\:\:\:\:\:\psi^{\dag} \mapsto \psi^{\dag}_n,\:\:\:\:\: \partial \psi \mapsto \frac{\psi_{n+1}-\psi_n}{\epsilon}\:\:\:\: \text{y} \:\:\:\: \psi^{\dag }\mapsto \frac{\psi^{\dag}_{n+1}-\psi^{\dag}_n}{\epsilon}.
\end{equation}
Luego, considerando el caso particular con $\epsilon=1$, resulta que
\begin{equation}
\mathcal{H}=\sum_n\left(-\frac{i}{2}(\psi_n^{\dag}\gamma^0\gamma^1(\psi_{n+1}-\psi_n)-\text{c.c.})+m\psi_n^{\dag}\gamma^0\psi_n-\mu \psi_n^{\dag}\psi_n\right),
\end{equation}
donde $\lbrace\psi_i,\psi^{\dag}_j\rbrace=\delta_{ij}$. Dado que el sistema posee invarianza traslacional, resulta conveniente  realizar una transformada de Fourier discreta del campo en cada sitio
\begin{equation}
\psi_n=\frac{1}{\sqrt{N}}\sum_k\varphi_k e^{ikn},
\end{equation}
donde $N$ es tal que $\sum_k e^{ikn}=N\delta_{N,0}$ (efectivamente $N$ resulta ser el número de sitios pensando en que todavía no hemos efectuado $N\rightarrow \infty$). Entonces, el hamiltoniano resulta
\begin{equation}
\mathcal{H}=\sum_k \varphi_k^{\dag}\left(\sin(k)\gamma^0\gamma^1+m\gamma^0-\mu \right)\varphi_k=\sum_k \varphi_k^{\dag}M(k)\varphi_k.
\end{equation}
Utilizando propiedades generales de las matrices gamma puede probarse que $\text{Tr}(M)=-2\mu$ y $\text{Det}(M)=-(m^2+\sin(k)^2-\mu^2)$, y los autovalores de $M(k)$ resultan
\begin{equation}
0=E_k^2-\text{Tr}(M)E_k+\text{Det}(M)\:\:\:\Leftrightarrow\:\:\: E_k=-\mu\pm \sqrt{m^2+\sin(k)^2}.
\end{equation}
Para calcular los autoespinores adoptaremos la representación dada por $\gamma^0=\sigma^1$ y $\gamma^1=i\sigma^2$ como se sugiere en \cite{Casini:2009sr}.
Luego, si denotamos a $u$ y $v$ a los autoespinores asociados a $E_k^+$ y $E_k^-$ respectivamente:

\begin{equation}
E_k^+ \Leftrightarrow u=a\begin{pmatrix}
m \\ \sin(k)+\sqrt{m^2+\sin(k)^2}
\end{pmatrix}\:\:\:\: \text{y}  \:\:\:\: E_k^- \Leftrightarrow v=b \begin{pmatrix}
m \\ \sin(k)-\sqrt{m^2+\sin(k)^2}
\end{pmatrix}.
\end{equation}
Las constantes de normalización $a$ y $b$ son extremadamente importantes\footnote{Quisiera agradecer a Raimel A. Medina Ramos por la observación y por los distintos aportes en los cálculos de este sistema.} ya que es necesario construir una matriz $U(k)$ constituida por los autoespinores. Mas aun, $U$ debe ser unitaria para preservar las relaciones de anticonmutación canónicas. Entonces:
\begin{equation}
U(k)=\left(u\: v\right)\:\:\:\: \text{y}\:\:\:\:  U^{\dag}(k)U(k)=\mathbb{I}\:\:\:\: 
 \text{implican}
\end{equation}
\begin{equation}
\begin{split}
a^2 & =\frac{1}{2\sqrt{m^2+\sin(k)^2}(\sqrt{m^2+\sin(k)^2}+\sin(k))}, \\ b^2 & =\frac{1}{2\sqrt{m^2+\sin(k)^2}(\sqrt{m^2+\sin(k)^2}-\sin(k))}.
\end{split}
\end{equation}
Ahora, el producto externo de los autoespinores resulta\footnote{Para simplicar los cálculos, puede utilizarse la identidad $m^2=(\sqrt{m^2+\sin(k)^2}-\sin(k))(\sqrt{m^2+\sin(k)^2}+\sin(k))$.} (definimos el producto externo de dos vectores como de costumbre $(x^{\dag}y)_{ij}=x^{\dag}_iy_j$):
\begin{equation}
u^{\dag}u=\begin{pmatrix}
\frac{1}{2}-\frac{\sin(k)}{2\sqrt{m^2+\sin(k)^2}} & \frac{m}{2\sqrt{m^2+\sin(k)^2}} \\ \frac{m}{2\sqrt{m^2+\sin(k)^2}} & \frac{1}{2}+\frac{\sin(k)}{2\sqrt{m^2+\sin(k)^2}}
\end{pmatrix} = \frac{1}{2}\mathbb{I}+\frac{m\gamma^0}{2\sqrt{m^2+\sin(k)^2}}+\frac{\sin(k)\gamma^0\gamma^1}{2\sqrt{m^2+\sin(k)^2}}
\end{equation}
\begin{equation}
v^{\dag}v=\begin{pmatrix}
\frac{1}{2}+\frac{\sin(k)}{2\sqrt{m^2+\sin(k)^2}} & -\frac{m}{2\sqrt{m^2+\sin(k)^2}} \\ -\frac{m}{2\sqrt{m^2+\sin(k)^2}} & \frac{1}{2}-\frac{\sin(k)}{2\sqrt{m^2+\sin(k)^2}}
\end{pmatrix} = \frac{1}{2}\mathbb{I}-\frac{m\gamma^0}{2\sqrt{m^2+\sin(k)^2}}-\frac{\sin(k)\gamma^0\gamma^1}{2\sqrt{m^2+\sin(k)^2}}
\end{equation}
Esperamos que estos resultados sean independientes de la representación utilizada, dado que los resultados físicos no puede depender de ello. Para computar el correlador, procedemos de forma similar a la sección (\ref{sec:EE_lattice_f}) dado que nuestra densidad hamiltoniana es de la forma $\mathcal{H}=\sum_{ij}H_{ij}c^{\dag}_ic_j$. La función de correlación de dos puntos va a depender de las autofunciones $\chi_k$ y de las autoenergías $E_k^{\pm}$ como:
\begin{equation}
C_{ij}\equiv \langle \psi^{\dag}_i \psi_j\rangle={\sum_k \chi^{\dag}_k(i)\chi_k(j)\frac{1}{1+e^{ E_k}}}\:\stackrel{T\rightarrow 0}{=}\:\sum_k \chi^{\dag}_k(i)\chi_k(j)\Theta(-E_k),
\end{equation}
donde $\Theta$ representa la función de Heaviside. Hasta el momento, hemos realizado dos cosas, primero realizamos una transformada de Fourier discreta y luego diagonalizamos $M(k)$. Además, en nuestro ejemplo, el producto entre autoespinores no devuelve un escalar, sino que por tratarse de un producto externo, retorna una matriz de $2\times 2$. Finalmente:
\begin{equation}
C_{ij}=\sum_k \frac{e^{ik(j-i)}}{N}\left(u^{\dag}(k)u(k)\Theta(-E_k^+(\mu))+v^{\dag}(k)v(k)\Theta(-E_k^-(\mu))\right)
\end{equation}
En el límite $N\rightarrow \infty$, la sumatoria se transforma en una integral $\sum_{k} \mapsto \int \frac{dk}{(\frac{2\pi}{N})}$ y el intervalo de integración se reacomoda $k \in (-\pi,\pi)$. Entonces, obtenemos la expresión para el correlador para $\mu$ arbitrario 
\begin{equation}
C_{ij}=\int_{-\pi}^{\pi} \frac{dk}{2\pi}  \left(u^{\dag}(k)u(k)\Theta(-E_k^+(\mu))+v^{\dag}(k)v(k)\Theta(-E_k^-(\mu))\right) e^{ik(j-i)}. 
\label{eq:corr_mu}
\end{equation}
En el caso particular en el que $\mu=0$, $C_{ij}^0\equiv C_{ij}(\mu=0)$ coincide con la ecuación (201) de \cite{Casini:2009sr} a menos del signo de $m$\footnote{El signo de $m$ resulta irrelevante, ya que el valor que realmente importa es el de $m^2$.},
\begin{equation}
\begin{split}
C_{ij}^0&=\int_{-\pi}^{\pi} \frac{dk}{2\pi} v^{\dag}(k)v(k)e^{ik(j-i)}\\
&=\int_{-\pi}^{\pi} \frac{dk}{2\pi}\left(\frac{1}{2}\mathbb{I}-\frac{m\gamma^0}{2\sqrt{m^2+\sin(k)^2}}-\frac{\sin(k)\gamma^0\gamma^1}{2\sqrt{m^2+\sin(k)^2}} \right)e^{ik(j-i)}\\
&=\frac{1}{2}\delta_{ij}+\int_{-\pi}^{\pi} \frac{dk}{4\pi}\frac{-m\gamma^0+\sin(k)\gamma^0\gamma^1}{\sqrt{m^2+\sin(k)^2}}e^{ik(i-j)}.
\end{split}
\label{eq:corr_mu_0}
\end{equation}
Un resultado interesante de (\ref{eq:corr_mu}) es que el correlador para $\mu=0$ (\ref{eq:corr_mu_0}) no se ve modificado al cambiar el potencial químico a menos que $\mu \geq m$. Por otro lado, si $\mu$ es tal que $\mu\geq \sqrt{m^2+1}$ resulta que $C_{ij}=\mathbb{I}\delta_{ij}$. Esto último es un efecto de la discretización del sistma puesto que se trabajó con un espaciado de $\epsilon=1$. Si se hubiese utilizado un $\epsilon$ genérico los nuevos resultados podrían obtenerse sencillamente mapeando: $k \mapsto k\epsilon$ y $\sin(\cdot)\mapsto \frac{\sin(\cdot)}{\epsilon}$. Luego, el efecto antes mencionado ocurriría sólo si $\mu \geq \sqrt{m^2+\frac{1}{\epsilon^2}}$, pero puesto que $\epsilon \rightarrow 0$ en el continuo, eso nunca podría cumplirse.\\

A partir de la expresión (\ref{eq:corr_mu}) se realizó un programa con el fin de calcular la entropía de entrelazamiento $S(R)$ de una región $V(R)=\{i,j:\: 1\leq i,j \leq R \}$ en función de $R$. Debe notarse que para emplear numéricamente la fórmula (\ref{eq:EE_matC}), se deben calcular los autovalores de la matriz $C$ restringida a la región $V(R)$. Sin embargo, cada elemento $C_{ij}$ es además una matriz de $2 \times 2$ debido al carácter espinorial del campo de Dirac. Por este motivo a la red original $(i \alpha)$ (con $\alpha=1,2$) se le asoció una nueva red $I=(i \alpha)$ definida de forma tal que\footnote{Si los índices originales estaban restringidos a $i,j=1,\ldots,R$, en la nueva red estarán restringidos a $I,J=1, \ldots,2R$.}
\begin{equation}
I=2(i-1)+\alpha.
\end{equation}
Mientras que la relación inversa está dada por
\begin{equation}
i=1+\left \lfloor \frac{I-1}{2}\right \rfloor,\:\:\:\:\:\: \alpha\equiv I \: (mod\:2).
\end{equation}
Para optimizar el programa, las componentes de los correladores las renombramos
\begin{equation}
C_{(i1) (j1)} \equiv A_+(j-i)\;,\;C_{(i1) (j2)}=C_{(i2) (j1)} \equiv B(j-i)\;,\;C_{(i2) (j2)}\equiv A_-(j-i)\,,
\end{equation}
y en la nueva red resulta que 
\begin{equation}
C_{IJ}=
\begin{cases}
A_+(\frac{J-I}{2})\:\:\:\: \text{si I,J}\equiv 0 \:(mod\:2) \\
A_-(\frac{J-I}{2})\:\:\:\: \text{si I,J}\equiv 1 \:(mod\:2) \\
B(\frac{J-I+(-1)^I}{2}) \:\:\:\: \text{caso contario}
\end{cases}.
\end{equation}
Pueden obtenerse expresiones un poco más trabajadas para $A_{\pm}$ y $B$ utilizando argumentos de paridad, resultando
\begin{equation}
\begin{split}
A_\pm(x)&= \frac{1}{2} \delta_{x,0}+\frac{1+(-1)^{x}}{2\pi}\,\frac{\sin (\Lambda x)}{x} \pm i \int_\Lambda^\pi\,\frac{dk}{\pi}\,\frac{\sin(kx)\:\sin (k)}{2 \sqrt{m^2+\sin^2(k) }}\\ &\:\:\:\:\:\:\:\:\:\:\:\:\:\:\:\:\:\:\:\:\:\:\:\:\:\:\:\:\:\:\:\:\:\pm i(-1)^{x}\int_0^\Lambda\,\frac{dk}{\pi}\,\frac{\sin(kx)\: \sin (k)}{2 \sqrt{m^2+\sin^2(k) }} \\
B(x) &= \int_\Lambda^\pi\,\frac{dk}{\pi}\,\frac{m\: \cos(kx)}{2 \sqrt{m^2+\sin^2(k) }}-(-1)^{x}\int_0^\Lambda\,\frac{dk}{\pi}\,\frac{m\: \cos(kx)}{2 \sqrt{m^2+\sin^2(k) }}\, ,
\end{split}
\end{equation}
y siendo el parámetro $\Lambda=\arcsin(\sqrt{\mu^2-m^2})$. A partir del programa\footnote{Dada la periodicidad de la relación de dispersión, se debe dividir por dos al valor de la entropía de entrelazamiento para obtener el resultado correcto asociado al campo de Dirac en el continuo.} se pudieron obtener gráficos para la entropía de entrelazamiento como el de (\ref{fig:EE_1+1}). También para dicho ejemplo, se ajustó una dependencia funcional de la forma $a+b\log(\frac{R}{\epsilon})$ con $\epsilon=1$ y se obtuvo un valor de $b\approx 0.343$ que se aproxima a $1/3$, valor de la carga central del álgebra de Virasoro, que aparece como una constante universal acompañando al término logarítmico de la entropía en sistemas en 1+1 dimensiones para $mR \gg 1$\cite{Casini_2007}. 
\begin{figure}[ht]
    \centering
    \includegraphics[width=0.6\linewidth]{lattice_femion_11_m.png}
    \caption{Simulación de la entropía de entrelazamiento para una región $R$ con $m=0.1$, $\mu=0.11$ y espaciado $\epsilon=1$. Se ajustó una dependencia de la forma $a+b\log(\frac{R}{\epsilon})$ con $\epsilon=1$ y se obtuvieron los valores: $a \approx 0.603$ y $b\approx 0.343$.}
    \label{fig:EE_1+1}
\end{figure}

La próxima tarea a realizarse como continuación del trabajo consistirá en obtener curvas como las de (\ref{fig:EE_1+1}), pero en el continuo de la teoría. Este procedimiento no es tan sencillo computacionalmente dado que hay dos parámetros adimensionales a controlar para un espaciado $\epsilon=1$ constante y para un dado $R$: $\lambda_m=m R$ y $\lambda_{\mu}=\mu R$. A priori, se esperaría que: para $\lambda_i \ll 1$ la entropía sea proporcional a $\log(R)$ por comportarse como una CFT \cite{Casini:2009sr}; para $\mu > m$ y $\lambda_{\mu} \gg 1$ domine la superficie de Fermi y la entropía sea también proporcional a $\log(R)$\cite{Swingle:2014}. Por tanto, a partir de la simulación se podría observar cómo es la transición entre un comportamiento y el otro al variar los parámetros adimensionales.
%%% Local Variables:
%%% mode: latex
%%% TeX-master: "template"
%%% End:

%\chapter{Conclusiones y próximos pasos}\label{cap5}
\chapterquote{Amigo Edison. Solo conozco a dos personas que merezcan llamarse “genio”. Una es usted; la otra es el joven que porta esta carta}{Carta de recomendación de Nikola Tesla.}

En el presente trabajo se estudiaron algunos aspectos fundamentales de la Teoría de Campos en el límite no relativista, y hacia el final, introduciendo una densidad finita de materia.

En el segundo capítulo se obtuvo una acción efectiva a bajas energías partiendo del campo relativista de Klein-Gordon. Con ella se determinaron las soluciones y la teoría se cuantizó siguiendo el esquema canónico. Se estudiaron estados de partículas de dos tipos: por un lado localizadas en el espacio de momentos  y por el otro, localizadas en el espacio real. Éste último tipo de partículas son inexistentes en la teoría relativista original. También se calculó el propagador a partir del formalismo operatorial, y se discutió la pérdida de la noción de microcausalidad debido a que la máxima velocidad de propagación no se encuentra acotada.

En el tercer capítulo, se estudió el Teorema de Reeh-Schlieder. Se analizó su demostración en el caso relativista, y en particular, el corolario que implica la separabilidad del vacío. Por otro lado, se extendió la demostración del teorema al límite no relativista, aunque en este caso la separabilidad del vacío no ocurría debido a la falta de conmutatividad de los campos a tiempos distintos. Además, se introdujo la noción de entropía de entrelazamiento y se concluyó que la misma es cero para cualquier región en el límite no relativista. La validez del teorema de Reeh-Schlider y la nulidad de la entropía de entrelazamiento a priori hubiesen parecido incompatibles, sin embargo la existencia de estados de partículas localizadas en posición y la existencia de traslaciones espacio-temporales en el límite no relativista, permiten que ambas proposiciones puedan coexistir.



Finalmente, en el cuarto capítulo se consideró un campo de Dirac libre discretizado en 1+1 dimensiones espacio-temporales con una densidad no trivial de materia. Se realizaron cálculos para determinar la función de correlación de dos puntos, que restringida a una región permite obtener la entropía de entrelazamiento. En particular se observó que para $|\mu|\leq m$ la entropía continua siendo la misma que para $\mu=0$. Por otro lado, se realizó un programa que por el momento calcula los valores de la entropía de entrelazamiento de un intervalo en la teoría discretizada. Para $m=0.1$ y $\mu=0.11$ se ajustó una dependencia logarítmica y se obtuvo una constante multiplicativa de $0,343$ que coincide aproximadamente con el término universal de $1/3$ que aparece en sistemas en 1+1 dimensiones. 

Los próximos pasos consistirán en obtener las curvas de entropía de entrelazamiento que sean representativas de la teoría en el continuo. En dicha tarea deben modificarse dos páramentros adimensionales, por tal motivo se requiere optimizar al máximo el código numérico  antes de emplearlo. Una vez obtenidas las curvas, se esperarían apreciar distintos regímenes, entre ellos uno en el cual la entropía sea constante como en el límite no relativista. También, se buscarán utilizar otros indicadores como la entropía relativa para cuantificar como el vacío va “volviendose separable” al prender un potencial químico. En un período posterior, el objetivo del trabajo consistirá en estudiar el mismo sistema pero en 2+1 dimensiones espacio-temporales. En tal caso, se espera la presencia de efectos más dramáticos debido a la existencia de la superficie de Fermi. Esto se debe a que en $d=3$ para una CFT, la entropía de una bola escalea con la ley de áreas, es decir $S(R)\sim \frac{R}{\epsilon}$ (contribución UV divergente), mientras que debido a la presencia del potencial químico, la entropía debería escalear como $S(R)\sim (k_F R) \log(k_F R)$ (contribución finita). Por lo tanto, esto conduciría a una violación logarítmica de la ley de áreas, efecto no apreciable en $d=2$.

%%% Local Variables:
%%% mode: latex
%%% TeX-master: "template"
%%% End:

%\include{cap6}
%\include{cap7}
%\include{cap8}

\appendix
%\chapter{Teorema de reflexi\'on de Schwarz}\label{app:schwarz}

\chapterquote{Eppur si muove.}{Galileo Galilei. Tras abjurar en 1633.}


%%%%%%%%%%%%%%%%%%%%%%%%%%%%%%%%%%%%%%%%%%%%%%%%%%%%%%%%%%%%%%%%%%%%%%%%
En relación con la demostración del teorema de Reeh-Schlieder probaremos que $g(u)\equiv 0\:\:\forall u\in \mathbb{C}$ si $g$ es holomorfa en el semiplano superior $\mathbb{H}$, continua a medida que $u$ se acerca al eje real por encima y nula en el segemento $I=(-\epsilon,\epsilon)$ del eje real. Por esa razón, invocamos al teorema de reflexión de Schwarz.\\

\textbf{Teorema de reflexión de Schwarz}\footnote{En la literatura, se encuentra el denominado \textit{edge of the wedge theorem} que es una especie de generalización del Teorema de reflexión de Schwarz para varias variables.} Sea $A\subset \mathbb{C}$ un conjunto abierto con la propiedad de que $z\in A\Leftrightarrow\bar{z}\in A$. Sea $f$ definida en $\{z\in A:\text{Im}(z)\geq 0\}$ y continua en dicho conjunto, asi como holomorfa en $\{z\in A:\text{Im}(z)>0\}$. Supongamos que $z\in A\cap \mathbb{R}$ implica que $f(z)$ sea real. Entonces $f$ posee una extensión analítica $\tilde{f}:A\rightarrow\mathbb{C}$ tal que
$$\tilde{f}(z)=\begin{cases}f(z)& \text{si } z\in A,\:\text{Im(z)}\geq 0 \\ \overline{f(\bar{z})}& \text{si } z\in A,\:\text{Im(z)}< 0\end{cases}.$$  
\\
 
Definiendo $A:=\{z\in \mathbb{C}: \text{ si }z\in \mathbb{R}\Rightarrow z\not \in(-\infty,-\epsilon]\cup[\epsilon,\infty)\}$ puede verse que $A$ es un conjunto abierto. Luego, como $g$ es holomorfa en el semiplano superior, continua a medida que u$u$ se acerca al eje real por encima y real en $I=(-\epsilon,\epsilon)$ (porque $g(z)=0$ si $z\in I$), debido al teorema de reflexión de Schwarz concluimos que existe una extensión holomorfa $\tilde{g}$ en $A$. En otras palabras, $\tilde{g}$ resulta holomorfa en el semiplano superior/inferior así como en I (en donde $\tilde{g}$ se anula). Ahora, debemos hacer uso de la holomorfía de $\tilde{g}$ y de su nulidad en $I$. Por ese motivo, utilizaremos el hecho de que las funciones holomorfas definidas en un domino (es decir, en un conjunto abierto y conexo) tienen una propiedad interesante en relación con sus ceros:\\

\textbf{Proposición}\footnote{Me gustaría agradecer a Javier Fernández por indicarme que esta proposición era necesaria para la demostración.} Sea $B\subset \mathbb{C}$ un dominio (un conjunto abierto y conexo) y $f$ una función  compleja holomorfa en $B$. Si $z_0\in B$ es un punto de acumulación de ceros de $f$, entonces $f \equiv 0$ en $B$.\\

Utilizando la proposición es evidente que $\tilde{g}$ debe anularse en $A$, porque cada $z\in I$ es un punto de acumulación de ceros en $\tilde{g}$. Finalmente, como sabemos que $g$ es continua a medida que $u$ se aproxima al eje real concluimos que $g(u)=0$ si $x\in \mathbb{R}$. $\square$

%%% Local Variables:
%%% mode: latex
%%% TeX-master: "template"
%%% End:

%\include{apend2}
%\include{apend3}

\begin{biblio}
\bibliography{mibib}
\end{biblio}


%\begin{postliminary}
\begin{seccion}{Agradecimientos}
%\chapterquote{I knew that good like bad becomes a routine, that the temporary tends to endure, that what is external permeates to the inside, and that the mask, given time, comes to be the face itself.}{Marguerite Yourcenar.}
%.
%Agradezco a A,B, y C
Inserte aquí su agradecimiento.

\end{seccion}
%\end{postliminary}

\end{document}

\grid
